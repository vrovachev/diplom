\documentclass[a4paper,14pt,russian]{extreport}

\hoffset 0pt
\voffset 0pt

\usepackage[
  a4paper, includehead, includefoot, mag=1000,
  headsep=0mm, headheight=0mm,
  left=25mm, right=15mm, top=20mm, bottom=20mm
]{geometry}

\usepackage[T2A]{fontenc}
\usepackage[utf8]{inputenc}
\usepackage[russian]{babel}

% \usepackage{cmap} % Улучшенный поиск русских слов в полученном pdf-файле
\usepackage[unicode, pdftex]{hyperref}
\usepackage{pdfpages}
\usepackage[nottoc]{tocbibind}

\usepackage[onehalfspacing]{setspace} %"умное" расстояние между строк - установить 1.5 интервала от нормального
\usepackage{cite}  %"умные" библиографические ссылки (сортировка и сжатие)
\usepackage{indentfirst} %делать отступ в начале параграфа
\usepackage{enumerate}  %создание и автоматическая нумерация списков
\usepackage{longtable} % Длинные таблицы
\usepackage{multirow,makecell,array} % Улучшенное форматирование таблиц
\usepackage{graphicx} \graphicspath{{images/}}
\usepackage{pdflscape} % Для включения альбомных страниц
\renewcommand{\rmdefault}{ftm} % Включаем Times New Roman
%%% Выравнивание и переносы %%%
\sloppy % Избавляемся от переполнений
\clubpenalty=10000 % Запрещаем разрыв страницы после первой строки абзаца
\widowpenalty=10000 % Запрещаем разрыв страницы после последней строки абзаца
\righthyphenmin=2 % Минимальное число символов при переносе - 2.

\usepackage{fancyvrb}

\usepackage{amssymb,amsmath,amsfonts,latexsym,mathtext} %расширенные наборы  математических символов

\usepackage{amsthm}
\theoremstyle{definition}
\newtheorem{theorem}{Теорема}
\newtheorem{proposition}[theorem]{Предложение}
\newtheorem{corollary}[theorem]{Следствие}
\newtheorem{lemma}[theorem]{Лемма}
\newtheorem{definition}[theorem]{Определение}
\newtheorem{example}[theorem]{Пример}
\newtheorem{remark}[theorem]{Замечание}

\usepackage[tableposition=top]{caption}
\DeclareCaptionLabelFormat{gostfigure}{Рисунок #2}
\DeclareCaptionLabelFormat{gosttable}{Таблица #2}
\DeclareCaptionLabelSeparator{gost}{~---~}
\captionsetup{labelsep=gost}
\captionsetup[figure]{labelformat=gostfigure}
\captionsetup[table]{labelformat=gosttable}

\usepackage{fancyhdr}

\pagestyle{fancyplain}
\renewcommand{\headrulewidth}{0pt}
\fancyhf{}
\rfoot{\fancyplain{}{\thepage}}

\makeatletter 
\renewcommand\chapter{\if@openright\cleardoublepage\else\clearpage\fi \thispagestyle{fancyplain}%
\global\@topnum\z@ \@afterindentfalse \secdef\@chapter\@schapter} 
\makeatother

\addtocontents{toc}{\protect\thispagestyle{fancyplain}}

\setcounter{page}{1}

\usepackage{titlesec}
\titleformat{\chapter}
	{\normalsize\bfseries}
	{\thechapter}
	{1em}{}
	
\titleformat{\section}
	{\normalsize\bfseries}
	{\thesection}
	{1em}{}
	
\titleformat{\subsection}
	{\normalsize\bfseries}
	{\thesubsection}
	{1em}{}

\titleformat{\paragraph}
	{\normalsize\bfseries}
	{\thesection}
	{1em}{}

	
\titlespacing*{\chapter}{0pt}{-30pt}{8pt}
\titlespacing*{\section}{\parindent}{*4}{*4}
\titlespacing*{\subsection}{\parindent}{*4}{*4}
\titlespacing*{\paragraph}{\parindent}{*4}{*4}

\addto\captionsrussian{%
  \renewcommand\contentsname{CОДЕРЖАНИЕ}
  \renewcommand\appendixname{ПРИЛОЖЕНИЕ}
  \renewcommand\bibname{СПИСОК ИСПОЛЬЗОВАННЫХ ИСТОЧНИКОВ}
}

\usepackage{enumitem}
\makeatletter
    \AddEnumerateCounter{\asbuk}{\@asbuk}{м)}
\makeatother
\setlist{nolistsep}
\renewcommand{\labelitemi}{-}
\renewcommand{\labelenumi}{\asbuk{enumi})}
\renewcommand{\labelenumii}{\arabic{enumii})}

\usepackage{tocloft}
\renewcommand{\cfttoctitlefont}{\hspace{0.38\textwidth} \bfseries\MakeUppercase}
\renewcommand{\cftbeforetoctitleskip}{-1em}
\renewcommand{\cftaftertoctitle}{\mbox{}\hfill \\ \mbox{}\hfill{\footnotesize Стр.}\vspace{-2.5em}}
\renewcommand{\cftchapfont}{\normalsize\bfseries}
\renewcommand{\cftsecfont}{\hspace{31pt}}
\renewcommand{\cftsubsecfont}{\hspace{11pt}}
\renewcommand{\cftbeforechapskip}{1em}
\renewcommand{\cftparskip}{-1mm}
\renewcommand{\cftdotsep}{1}
\renewcommand{\cftchapdotsep}{\cftdotsep}
\setcounter{tocdepth}{2} % задать глубину оглавления — до subsection включительно

\newcommand{\likechapterheading}[1]{
    \newpage
    \begin{center}
    \textbf{\MakeUppercase{#1}}
    \end{center}}

\newcommand{\abbreviations}{\likechapterheading{ОБОЗНАЧЕНИЯ И СОКРАЩЕНИЯ}\addcontentsline{toc}{chapter}{ОБОЗНАЧЕНИЯ И СОКРАЩЕНИЯ}}
\newcommand{\definitions}{\likechapterheading{ОПРЕДЕЛЕНИЯ}\addcontentsline{toc}{chapter}{ОПРЕДЕЛЕНИЯ}}
\newcommand{\abbrevdef}{\likechapterheading{ОПРЕДЕЛЕНИЯ, ОБОЗНАЧЕНИЯ И СОКРАЩЕНИЯ}\addcontentsline{toc}{chapter}{ОПРЕДЕЛЕНИЯ, ОБОЗНАЧЕНИЯ И СОКРАЩЕНИЯ}}
\newcommand{\intro}{\likechapterheading{ВВЕДЕНИЕ}\addcontentsline{toc}{chapter}{ВВЕДЕНИЕ}}
\newcommand{\conclusions}{\likechapterheading{ЗАКЛЮЧЕНИЕ}\addcontentsline{toc}{chapter}{ЗАКЛЮЧЕНИЕ}}

\makeatletter
  \renewcommand{\@biblabel}[1]{#1.}	% Заменяем библиографию с квадратных скобок на точку:
\makeatother

\newcommand{\biblio}{
  \bibliographystyle{ugost2008}	% Оформляем библиографию в соответствии с ГОСТ 7.0.5
  \nocite{*}
  \bibliography{biblio}
}

\newcommand{\appendxchapter}[1]{ 
    \clearpage
    \stepcounter{chapter}    
    \chapter*{\appendixname~\Asbuk{chapter}\;#1}
    \addcontentsline{toc}{chapter}{\appendixname~\Asbuk{chapter}\;#1}
}
 

%\graphicspath{{Graphs/}{e_0/}{e_0.25/}{e_0.5/}{medium_reorientation/}{large_reorientation/}%{variant_1/}{variant_2/}{variant_3/}}

\renewcommand{\rmdefault}{cmr} % Шрифт с засечками
\renewcommand{\sfdefault}{cmss} % Шрифт без засечек
\renewcommand{\ttdefault}{cmtt} % Моноширинный шрифт

\begin{document}

\includepdf[pages={1}]{titulDiplom.pdf}

\tableofcontents

\abbrevdef
Дифференциальное уравнение — уравнение, связывающее значение производной функции с самой функцией, значениями независимой переменной, числами (параметрами). Дифференциальные уравнения – это соотношение вида $ F(x_1, x_2, x_3,...,y, y', y'',...,y^{(n)}) = 0 $, связывающее независимые переменные $ x_1, x_2, x_3,... $ функцию $ y $ этих независимых переменных и ее производные до n-го порядка. При этом функция $ F $ определена и достаточное число раз дифференцируема в некоторой области изменения своих аргументов. \\

Дифференциальный оператор - оператор, определённый некоторым дифференциальным выражением и действующий в пространствах функций на дифференцируемых многообразиях, или в пространствах, сопряжённых к пространствам этого типа. Дифференциальные операторы представляют собой обобщение операции дифференцирования. Простейший дифференциальный оператор $ D $, действуя на функцию $ y $, "возвращает" первую производную этой функции: $ Dy(x) = y'(x) $. \\

Гладкая функция - функция, все частные производные которой до порядка $ n $ включительно существуют и непрерывны. Это означает “гладкость” порядка $ n $. \\

$ f \in C[0,1] $ - функция, непрерывная на отрезке $[0,1]$. \\

$ f \in C^m[0,1] $ - функция, гладкая на отрезке $[0,1]$. \\

$ \Vert x \Vert_{C[0,1]} = \max\limits_{t \in [0,1]}\vert x(t) \vert $. Эту норму также называют нормой Чебышёва или равномерной нормой, так как сходимость по этой норме эквивалентна равномерной сходимости. \\

$ \Vert x \Vert_{L_2[0,1]} = \sqrt{\int\limits_0^x f^2(t)dt} $. В $ L_2 $ норма порождается скалярным произведением. Таким образом, вместе с понятием "длины" здесь имеет смысл и понятие "угла", а следовательно и смежные понятия, такие как ортогональность, проекция. \\

$ \Vert f \Vert_{L_\infty[0,1]} = ess\sup\limits_{x \in [0,1]} \vert f(x) \vert $. Где $ ess\sup $ - существенный супремум функции. \\
 
\label{theorem Banach Steinhaus}
Теорема Банаха-Штейнгауса. Пусть $ {A_n} $ - последовательность линейных операторов, $ A_0 $ - некоторый фиксированный линейный оператор в банаховом пространстве. Для того, чтобы  последовательность $ {A_n} $ линейных операторов поточечно сходилось к оператору $ A_0 $ или $ A_nx \rightarrow A_0x \forall x \in X $ необходимо и достаточно, чтобы:
1) Последовательность норм $ \lbrace\Vert A_n  \Vert\lbrace $ была ограничена. ($ \lbrace\Vert A_n  \Vert\lbrace \leq C, C $ не зависит от $ n $)
2) $ A_nx \rightarrow A_0x $ для некоторого всюду плотного $ M \subset X $. \\

\label{theorem Weierstrass}
Аппроксимационная теорема Вейерштрасса.
Для любой непрерывной функции на отрезке можно подобрать последовательность многочленов, равномерно сходящихся к этой функции на отрезке.
Пусть $ f $ - непрерывная функция, определённая на отрезке $ [a,b] $. Тогда для любого $ \varepsilon > 0 $ существует такой многочлен $ p $ с вещественными коэффициентами, что для любого $ x $ из $ [a,b] $ выполнено условие $ \vert f(x) - p(x) \vert < \varepsilon $.
Если $ f(x) $ непрерывна на круге (периодична), то утверждение верно и для тригонометрических многочленов.
Теорема справедлива и для комплекснозначных функций, но тогда коэффициенты полинома $ p $ следует считать комплексными числами.

\intro
Такие вопросы теории дифференциальных уравнений, как существование, единственность, регулярность, непрерывная зависимость решений от начальных данных или правой части, явный вид решения дифференциального уравнения, определённого данным дифференциальным выражением, естественно интерпретируются в терминах теории операторов как задачи дифференциального оператора, определённого данным дифференциальным выражением в подходящих функциональных пространствах, а именно - как задачи о ядре, образе, изучении структуры области определения данного дифференциального оператора   или его расширения, непрерывности обратного оператора к данному дифференциальному оператору и явного построения этого обратного оператора.

Теория дифференциальных операторов позволяет разрешить ряд трудностей классической теории дифференциальных уравнений. Использование различных расширений обычных дифференциальных операторов приводит к понятию обобщённого решения соответствующего дифференциального уравнения (которое в ряде случаев, связанных, например, с эллиптическими задачами, оказывается необходимо классическим), а использование линейной структуры позволяет вводить понятие слабых решений дифференциальных уравнений. При выборе подходящего расширения дифференциального оператора, определённого дифференциальным выражением, важную роль играют связанные с конкретным видом последнего априорные оценки для решений, которые позволяют указать такие функциональные пространства, что в этих пространствах дифференциальных операторов непрерывен или ограничен.

Теория дифференциальных операторов даст возможность поставить и решить и ряд принципиально новых задач по сравнению с классическими задачами теории дифференциальных уравнений. Так, для нелинейных операторов представляют интерес изучение структуры множества его неподвижных точек и действие оператора в их окрестности, а также классификация этих особых точек и вопрос об устойчивости типа особой точки при возмущении данного дифференциального оператора; для линейных дифференциальных операторов кроме указанных выше задач, представляют интерес задачи об описании и изучении спектра дифференциальных операторов, построения его резольвенты, вычислений индекса, описание структуры инвариантных подпространств данного дифференциального оператора, построение связанного с данным дифференциальным оператором гармонического анализа (в частности, разложения по собственным функциям, что требует предварительного изучения вопросов полноты системы собственных и присоединённых функций), изучения линейных и нелинейных возмущений данного дифференциального оператора.

В данной работе рассматриваются второстепенные задачи построения резольвенты, нахождения  приближающих свойств резольвенты дифференциального оператора, расширения приближающих свойств резольвенты на пространство гладких функций, нахождения приближающих свойств производной от резольвенты порядка $ m $ в пространствах гладких функций \cite{Schtraus:1954}. 

А также рассматриваются основные задачи применения приближающих свойств резольвенты и приближающих свойств производной от резольвенты порядка $ m $ в пространствах гладких функций задачи для восстановления функции а также задачи восстановления производной функции порядка $ m $.


\chapter{Приближающие свойства резольвенты оператора \\ $ L_1:y', y(0)=0 $ на отрезке $ [\varepsilon, 1] $.}
Рассмотрим простейший дифференциальный оператор первого порядка $ L_1:y^{'}, y(0)=0 $. Обозначим через $ R_\lambda(L_1) $ его резольвенту, т.е. оператор $ R_\lambda(L_1)=(L_1-\lambda E)^{-1} $, где $ E $ - единичный оператор, $ \lambda $ - спектральный параметр (числовой параметр,  вообще говоря, комплексный). Найдем формулу для резольвенты.

\section{Лемма 1.1. Формула резольвенты дифференциального оператор первого порядка.}
\label{lemma1.1}

\textit{Для $ y(x) = R_\lambda(L_1)u$ имеет место формула:}

\begin{equation}
\begin{array}{c}

y(x) \equiv R_\lambda(L_1)u = \int\limits_0^x e^{\lambda(x-t)}u(t)dt.

\end{array}
\end{equation}

\textbf{Доказательство.} Пусть $ y = R_\lambda(L_1)u$. Тогда

\begin{equation}
\begin{array}{c}

y'- \lambda y = u,

\end{array}
\end{equation}

\begin{equation}
\begin{array}{c}

y(0)=0.

\end{array}
\end{equation}

По методу вариации произвольной постоянной общее решение уравнения (1.2) есть

\begin{equation}
\begin{array}{c}

y(x)=Ce^{\lambda x} + \int\limits_0^x e^{\lambda(x-t)}u(t)dt,

\end{array}
\end{equation}

Где $ C $ - произвольная постоянная. Находим эту постоянную из условия (1.3). Получаем $ C = 0 $. Отсюда приходим к формуле (1.1).
	
Положим в (1.1) $ \lambda \ -r $, где $ r > 0 $, и рассмотрим операторы $ rR_{-r}(L_1)$. Очевидно, эти операторы имеют вид:

\begin{equation}
\begin{array}{c}

rR_{-r}(L_1)u = r \int\limits_0^x e^{-r(x-t)}u(t)dt

\end{array}
\end{equation}

Выясним приближающие свойства операторов (1.5) при $ r \rightarrow \infty$.

\section{Лемма 1.2. Приближающие свойства резольвенты дифференциального оператор первого порядка.}
\label{lemma1.2}

\textit{Для любой функции $ u(x) \in C[0,1] $ имеет место сходимость:}

\begin{equation}
\begin{array}{c}

\Vert rR_{-r}(L_1)u-u \Vert _{C[\varepsilon ,1]} \rightarrow 0 $ \textit{при} $ r \rightarrow \infty,

\end{array}
\end{equation}

\textit{ $ \varepsilon $ - произвольное малое положительное число.}

\textbf{Доказательство.} Пусть сначала $ u(x) \in C^1[0,1] $. Тогда

\begin{equation}
\begin{array}{c}
\nonumber

\int\limits_0^x e^{-r(x-t)}u(t)dt = \frac{1}{r}\bigl\vert_0^x e^{-r(x-t)}u(t) - \frac{1}{r}\int\limits_0^x e^{-r(x-t)}u'(t)dt = \\
= \frac{1}{r} \biggl[u(x)-e^{-rx}u(0)-\int\limits_0^x e^{-r(x-t)}u'(t)dt\biggl].

\end{array}
\end{equation}

Отсюда получаем

\begin{equation}
\begin{array}{c}
\nonumber

rR_{-r}(L_1)u = u(x) - e^{-rx}u(0) - \frac{1}{r} (rR_{-r}(L_1)u').

\end{array}
\end{equation}

Тогда

\begin{equation}
\begin{array}{c}
\nonumber

rR_{-r}(L_1)u - u = u(x) - e^{-rx}u(0) - \int\limits_0^x e^{-r(x-t)}u'(t)dt.

\end{array}
\end{equation}

Далее,

\begin{equation}
\begin{array}{c}
\nonumber

\biggl| \int\limits_0^x e^{-r(x-t)}u'(t)dt \biggr| \leq \Vert u' \Vert_{C[0,1]} \int\limits_0^x e^{-r(x-t)}dt = \\
= \frac{1}{r} \Vert u' \Vert_{C[0,1]}.
(1 - e^{-rx}) 
\end{array}
\end{equation}

Тогда

\begin{equation}
\begin{array}{c}

\bigl| rR_{-r}(L_1)u - u \bigr| \leq e^{-rx} \bigl| u(0) \bigr| + \dfrac{1}{r} (1-e^{-rx})\Vert u' \Vert_{C[0,1]}.

\end{array}
\end{equation}

В силу того, что первое слагаемое в правой части последней оценки при $ x = 0 $ является константой, не зависящей от $ r $, то на всем отрезке $ [0,1] $ сходимости функций $ rR_{-r}(L_1)u $ к $ u(x) $ при $ r \rightarrow \infty $ мы отсюда не получим. Но если мы рассмотрим отрезок $ [\varepsilon, 1] $, где $ \varepsilon > 0 $ - любое фиксированное как угодно малое число, то тогда $ \Vert e^{-rx}_{C[\varepsilon,1]} = e^{-r\varepsilon} \rightarrow 0 \Vert $ при $ r \rightarrow \infty $. Отсюда следует утверждение леммы для $ u \in C^1[0,1] $.

Пусть теперь $ u(x) \in C[0,1] $. Покажем, что нормы операторов $ eR_{-r}(L_1) $, рассматриваемых как операторы из $  C[0,1] $ в $ C[\varepsilon,1] $, ограничены константой, не зависящей от $ r $.

Действительно, имеем:

\begin{equation}
\begin{array}{c}

\Vert rR_{-r}(L_1)u \Vert_{C[\varepsilon,1]} \leq \Vert rR_{-r}(L_1)u \Vert_{C[0,1]} = \\
= \Bigl\Vert r\int\limits_0^x e^{-r(x-t)}u(t)dt\Bigr\Vert_{C[0,1]} \leq \Vert u\Vert_{C[0,1]}\Vert 1 - e^{-rx}\Vert_{C[0,1]} \leq \Vert u \Vert_{C[0,1]}.

\end{array}
\end{equation}

Далее, множество функций $ u \in C^1[0,1] $ является всюду плотным в пространстве $ C[0,1] $. (Это следует из теоремы Вейерштрасса об аппроксимации непрерывной функции полиномами). Поэтому по теореме Банаха- Штейнгауса~\eqref{theorem banach steinhaus} из ограниченности норм операторов $ rR_{-r}(L_1) $ отсюда следует сходимость (1.6) для любой $ u \in C[0,1] $.

Лемма доказана.

Покажем теперь, что приближающие свойства операторов $ rR_{-r}(L_1) $ сохраняются и в пространствах гладких функций, т.е. в пространствах $ C^l[0,1] $.

Пусть сначала $ u \in C^{l-1}[0,1] $. Рассмотрим операторы $ D^kR_{-r}(L_1) \equiv (R_{-r}(L_1)u)_x^{(k)} k = 1,...,l, D^1 \equiv D (Du = u')$.

\section{Лемма 1.3. Формула резольвенты в пространствах гладких функций}
\label{lemma1.3}

\textit{Операторы $ D^kR_{-r}(L_1) $ имеют вид:}

\begin{equation}
\begin{array}{c}

D^kR_{-r}(L_1)u = u^{(k-1)}(x) - ru^{(k-2)}(x) + r^2u^{(k-3)}(x) - ... + \\
+ (-1)^{k-1}r^{k-1}u(x) + (-1)^kr^k\int\limits_0^x e^{-r(x-t)}u(t)dt, k = 1,...,l.

\end{array}
\end{equation}

\textbf{Доказательство.} Для $ k = 1 $ имеем:

\begin{equation}
\begin{array}{c}
\nonumber

DR_{-r}(L_1)u = (DR_{-r}(L_1)u)_x' = \Bigl( \int\limits_0^x e^{-r(x-t)}u(t)dt \Bigr)_x' = \\
= u(x) -r\int\limits_0^x e^{-r(x-t)}u(t)dt.

\end{array}
\end{equation}

Применяем метод математической индукции. Пусть (1.9) выполняется для $ k = m - 1 $, т.е.

\begin{equation}
\begin{array}{c}
\nonumber

D^{m-1}R_{-r}(L_1)u = u^{(m-2)}(x) - ru^{(m-3)}(x) + ... + \\
+ (-1)^{m-2}r^{m-2}u(x) + (-1)^{m-1}r^{m-1}\int\limits_0^x e^{-r(x-t)}u(t)dt.

\end{array}
\end{equation}

Тогда для $ k = m $ получим:

\begin{equation}
\begin{array}{c}
\nonumber

D^{m}R_{-r}(L_1)u = D(D^{m-1}R_{-r}(L_1)u) = u^{(m-1)}(x) - ru^{(m-2)}(x) + ... + \\
+ (-1)^{m-2}r^{m-2}u'(x) + (-1)^{m-1}r^{m-1}u(x) + (-1)^mr^m\int\limits_0^x e^{-r(x-t)}u(t)dt.

\end{array}
\end{equation}

Лемма доказана.

\section{Лемма 1.4. Приближающие свойства резольвенты в пространствах гладких функций.}
\label{lemma1.4}

\textit{Если $ u(x) \in C^l[0,1] $, то имеет место сходимость:}

\begin{equation}
\begin{array}{c}

\Vert rD^k R_{-r}(L_1)u -u^{(k)}(x) \Vert_{C[\varepsilon ,1]} \rightarrow 0 $ \textit{при} $ r \rightarrow \infty, k = 1,...,l.

\end{array}
\end{equation}

\textit{ $ \varepsilon $ - произвольное малое положительное число.}

\textbf{Доказательство.} Пусть $ k = 1 $. По лемме 1.3 имеем:

\begin{equation}
\begin{array}{c}

DR_{-r}(L_1)u = u(x) - r\int\limits_0^x e^{-r(x-t)}u(t)dt.

\end{array}
\end{equation}

Далее,

\begin{equation}
\begin{array}{c}

\int\limits_0^x e^{-r(x-t)}u(t)dt = e^{-rx}\int\limits_0^x e^{rt}u(t)dt = \\
= e^{-rx} \biggl\vert_0^x \frac{1}{r} e^{rt}u(t) - \frac{1}{r}\int\limits_0^x e^{-r(x-t)}u'(t)dt = \\
= \frac{1}{r}u(x) - \frac{1}{r} e^{-rx}u(0) - \frac{1}{r}\int\limits_0^x e^{-r(x-t)}u'(t)dt

\end{array}
\end{equation}

Подставляя (1.12) в (1.11), получим:

\begin{equation}
\begin{array}{c}
\nonumber

DR_{-r}(L_1)u = e^{-rx}u(0) + \int\limits_0^x e^{-r(x-t)}u'(t)dt

\end{array}
\end{equation}	
или
\begin{equation}
\begin{array}{c}

DR_{-r}(L_1)u = R_{-r}(L_1)u' + e^{-rx}u(0).

\end{array}
\end{equation}

Получим аналогичное (1.13) выражение для $ D^kR_{-r}(L_1) $ при $ k > 1 $. 

В силу (1.13) имеем:

\begin{equation}
\begin{array}{c}
\nonumber

D^2R_{-r}(L_1)u = D(DR_{-r}(L_1)u) = D[R_{-r}(L_1)u' + e^{-rx}u(0)] = \\
= DR_{-r}(L_1)u' - re^{-rx}u(0).

\end{array}
\end{equation}

Опять применяем (1.13), заменив $ u $ на $ u' $.

Получаем:

\begin{equation}
\begin{array}{c}
\nonumber

D^2R_{-r}(L_1)u = R_{-r}(L_1)u'' + e^{-rx}u'(0) - re^{-rx}u(0).

\end{array}
\end{equation}

Покажем, что в общем случае для любого $ k $ справедлива формула:

\begin{equation}
\begin{array}{c}

D^kR_{-r}(L_1)u = R_{-r}(L_1)u^{(k)} + e^{-rx}u^{(k-1)}(0) -er^{-rx}u^{(k-2)}(0) + ... + \\
+ (-1)^{k-1}r^{k-1}e^{-rx}u(0).

\end{array}
\end{equation}

Мы показали, что (1.14) выполняется для $ k = 1,2 $. Пусть это соотношение выполняется для $ k = m - 1 $, т.е.

\begin{equation}
\begin{array}{c}
\nonumber

D^{m-1}R_{-r}(L_1)u = R_{-r}(L_1)u^{(m-1)} + e^{-rx}u^{(m-2)}(0) - re^{-rx}u^{(m-3)}(0) + ... + \\
+ (-1)^{m-2}r^{m-2}e^{-rx}u(0).

\end{array}
\end{equation}

Тогда
\begin{equation}
\begin{array}{c}
\nonumber

D^mR_{-r}(L_1)u = D(D^{m-1}R_{-r}(L_1)u) = DR_{-r}(L_1)u^{(m-1)} + \\
+ D\bigl[e^{-rx}u^{(m-2)}(0) - re^{-rx}u^{(m-3)}(0) + ... + (-1)^{m-2}r^{m-2}e^{-rx}u(0)\bigr].

\end{array}
\end{equation}

Используя (1.13) с заменой $ u(x) $ на $ u^{(m-1)}(x) $, придём к выражению:
\begin{equation}
\begin{array}{c}
\nonumber

D^mR_{-r}(L_1)u = R_{-r}(L_1)u^{(m)} + e^{-rx}u^{(m-1)}(0) - re^{-rx}u^{(m-2)}(0) + \\
+ r^2e^{-rx}u^{(m-3)}(0) + ... + (-1)^{m-1}r^{m-1}e^{-rx}u(0).

\end{array}
\end{equation}

Из формулы (1.14) получаем оценку:
\begin{equation}
\begin{array}{c}
\nonumber

\bigl\Vert rD^kR_{-r}(L_1)u - u^{(k)}\bigr\Vert_{C[\varepsilon ,1]} \leq \bigl\Vert rR_{-r}(L_1)u^{(k)} - u^{(k)}\bigr\Vert_{C[\varepsilon ,1]} + e^{-r\varepsilon}\sum\limits_{j=0}^{k-1} r^j\vert u^{(k-j-1)}(0)\vert,

\end{array}
\end{equation}

И тогда соотношение (1.10) вытекает из леммы 1.2.

\textbf{Замечание}. Если в лемме 1.2~\eqref{lemma1.2} $ u(0) = 0 $, то сходимость (1.6) будет выполняться при $ \varepsilon = 0$, т.е. на всем отрезке $ [0,1] $. Если в лемме 1.4~\eqref{lemma1.4} $ u^{(m)}(0) = 0, m = 0,1,...,k $, , то сходимость 1.10 будет выполняться также при $ \varepsilon = 0 $.

В дальнейшем нам потребуются свойства не только самих операторов $ rR_{-r}(L_1) $, но и их степеней – операторов $ (rR_{-r}(L_1))^k $.

\section{Лемма 1.5. Формула степеней резольвенты}
\label{lemma1.5}

\textit{Операторы $ (rR_{-r}(L_1))^k $ имеют вид:}

\begin{equation}
\begin{array}{c}

(rR_{-r}(L_1))^ku = r^k\int\limits_0^x \dfrac{(x-t)^{k-1}}{(k-1)!}e^{-r(x-t)}u(t)dt.

\end{array}
\end{equation}

\textbf{Доказательство.} Обозначим для краткости $ rR_{-r}(L_1) = \Omega_{1r} $. Тогда для $ k = 2 $ имеем:

\begin{equation}
\begin{array}{c}
\nonumber

\Omega_{1r}^2u = r^2\int\limits_0^x e^{-r(x-t)} \int\limits_0^t e^{-r(x-t)} u(\tau)d\tau dt = \\
= r^2e^{-rx}\int\limits_0^x\int\limits_0^t e^{r\tau}u(\tau)d\tau dt = r^2e^{-rx}\int\limits_0^x\int\limits_\tau^x dte^{r\tau}u(\tau)d\tau = \\
= r^2e^{-rx}\int\limits_0^x (x-\tau)e^{r\tau}u(\tau)d\tau = r^2\int\limits_0^x (x-\tau)e^{-r(x-t)}u(\tau)d\tau .

\end{array}
\end{equation}

Заменим обозначение $ \tau $ на $ t $, получим (1.15) при $ k = 2 $. 

Пусть для $ k = m - 1 $ выполняется формула (1.15), т.е.

\begin{equation}
\begin{array}{c}
\nonumber

\Omega_{1r}^{m-1}u = r^{m-1}\int\limits_0^x\dfrac{(x-t)^{m-2}}{(m-2)!}e^{-r(x-t)}u(t)dt.

\end{array}
\end{equation}

Тогда
\begin{equation}
\begin{array}{c}
\nonumber

\Omega_{1r}^mu = \Omega_{1r}(\Omega_{1r}^{m-1}u) = r^m\int\limits_0^x e^{-r(x-t)} \int\limits_0^t\dfrac{(t-\tau)^{m-2}}{(m-2)!}e^{-r(t-\tau)}u(\tau)d\tau dt = \\
= r^me^{-rx}\int\limits_0^x\int\limits_0^t\dfrac{(t-\tau)^{m-2}}{(m-2)!}e^{r\tau}u(\tau)d\tau dt = \\
= r^me^{-rx}\int\limits_0^x\int\limits_\tau^x\dfrac{(t-\tau)^{m-2}}{(m-2)!}dte^{r\tau}u(\tau)d\tau = \\
= r^m\int\limits_0^x\dfrac{(x-\tau)^{m-1}}{(m-1)!}e^{-r(x-t)}u(\tau)d\tau,

\end{array}
\end{equation}

что и требовалось доказать.
\section{Лемма 1.6. Формула степеней резольвенты в пространстве непрерывно дифференцируемых функций первого порядка}
\label{lemma1.6}
\textit{Если $ u \in C^1[0,1] $, то операторы $ \Omega_{1r}^k $ имеют представление:}

\begin{equation}
\begin{array}{c}

\Omega_{1r}^ku = -\dfrac{r^{k-1}x^{k-1}e^{-rx}}{(k-1)!}u(0) + \Omega_{1r}^{k-1}u - \dfrac{1}{r}\Omega_{1r}^ku',

\end{array}
\end{equation}

\textit{где $ k \geq 2 $.}

\textbf{Доказательство.} Пусть $ k = 2 $.  Тогда из (1.15) мы получаем:

\begin{equation}
\begin{array}{c}
\nonumber

\Omega_{1r}^2u = r^2\int\limits_0^x (x-t)e^{-r(x-t)}u(t)dt.

\end{array}
\end{equation}
Интегрируем по частям:
\begin{equation}
\begin{array}{c}
\nonumber

\Omega_{1r}^2u = r^2\biggl\lbrace \dfrac{1}{r} \Big\vert_0^x e^{-r(x-t)}(x-t)u(t) - \dfrac{1}{r} \int\limits_0^x e^{-r(x-t)}[-u(t) + (x - t)u'(t)]dt \biggr\rbrace = \\
= r \biggl[ -xe^{-rx}u(0) + \int\limits_0^x e^{-r(x-t)}u(t)dt - \int\limits_0^x e^{-r(x-t)}(x-t)u'(t)dt \biggr] = \\
= -rxe^{-rx}u(0) + \Omega_{1r}u - \dfrac{1}{r}\Omega_{1r}^2u'.

\end{array}
\end{equation}

Предположим, что (1.16) выполняется для $ k = m - 1, m > 2 $. Тогда

\begin{equation}
\begin{array}{c}
\nonumber

\Omega_{1r}^{m-1}u = -\dfrac{r^{m-2}x^{m-2}e^{-rx}}{(m-2)!}u(0) + \Omega_{1r}^{m-2}u - \dfrac{1}{r}\Omega_{1r}^{m-1}u.

\end{array}
\end{equation}

Отсюда
\begin{equation}
\begin{array}{c}
\nonumber

\Omega_{1r}^mu = r\int\limits_0^x e^{r(x-t)}\biggl[ -\dfrac{r^{m-2}t^{m-2}e^{-rt}}{(m-2)!}u(0) \biggr]dt +  \Omega_{1r}(\Omega_{1r}^{m-2}u) - \\
- \dfrac{1}{r}\Omega_{1r}(\Omega_{1r}^{m-1}u') = -\dfrac{r^{m-1}x^{m-1}e^{-rx}}{(m-1)!}u(0) + \Omega_{1r}^{m-1}u - \dfrac{1}{r}\Omega_{1r}^mu',

\end{array}
\end{equation}

что и требовалось доказать.

\section{Лемма 1.7. Приближающие свойства резольвенты в пространствах непрерывных функций.}
\label{lemma1.7}

\textit{Для $ u(x) \in C[0,1] $ справедливы соотношения:}

\begin{equation}
\begin{array}{c}

\Vert \Omega_{1r}^ku - u \Vert_{C[\varepsilon ,1]} \rightarrow 0 $ \textit{при} $ r \rightarrow \infty , k = 1,2,...

\end{array}
\end{equation}

\textbf{Доказательство.} Для $ k = 1 $ соотношение (1.17) доказано в Лемме 1.2~\eqref{lemma1.2}.
Пусть $ k \geq 2 $, а $ u(x) \in C^k[0,1] $ обозначим $ \varphi_l(r,x) = - \dfrac{r^lx^le^{-rx}}{l!} $.
Из (1.16) имеем:
\begin{equation}
\begin{array}{c}
\nonumber

\Vert \Omega_{1r}^ku - u \Vert_{C[\varepsilon ,1]} \leq \Vert \varphi_{k-1}(r,x)u(0) \Vert_{C[\varepsilon, 1]} + \Vert \Omega_{1r}^{k-1}u - u \Vert_{C[\varepsilon ,1]} + \\
+ \biggl\Vert \dfrac{1}{r}\Omega_{1r}^ku' \biggr\Vert_{C[\varepsilon, 1]} \leq \Vert \varphi_{k-1}(r,x)u(0) \Vert_{C[\varepsilon, 1]} + \Vert \varphi_{k-2}(r,x)u(0) \Vert_{C[\varepsilon, 1]} + ... + \\
+ \Vert \varphi_1(r,x)u(0) \Vert_{C[\varepsilon, 1]} + \Vert \Omega_{1r}u - u \Vert_{C[\varepsilon ,1]} + \biggl\Vert \dfrac{1}{r}\Omega_{1r}^2u' \biggr\Vert_{C[\varepsilon, 1]} + \\
+ \biggl\Vert \dfrac{1}{r}\Omega_{1r}^3u' \biggr\Vert_{C[\varepsilon, 1]} + ... + \biggl\Vert \dfrac{1}{r}\Omega_{1r}^ku' \biggr\Vert_{C[\varepsilon, 1]}.
\end{array}
\end{equation}

Далее, поскольку $ \varphi_l(r,x) \leq r^le^{-r\varepsilon} $ на отрезке $ [\varepsilon ,1] $, то сумма слагаемых, содержащих функции $ \varphi_l(r,x), l = 1,...,k-1  $, имеет оценку $ O(r^{k-1}e^{-r\varepsilon}\Vert u \Vert_{C[0,1]}) $.

По лемме 1.2~\eqref{lemma1.2} $ \Vert \Omega_{1r}u - u \Vert_{C[\varepsilon ,1]} \rightarrow 0 $ при $ r \rightarrow \infty $ для любой $ u(x) \in C[0,1] $. 

Осталось показать, что слагаемые, содержащие $ u'(x) $, могут быть как угодно малыми при $ r \rightarrow \infty $, если $ u(x) \in C^k[0,1] $, т.е., что $ \bigl\Vert \dfrac{1}{r}\Omega_{1r}^lu' \bigr\Vert_{C[\varepsilon ,1]} \rightarrow 0$ при $ r \rightarrow \infty $ для $ l = 2,...,k $.

Пусть $ l = 2 $. Тогда из (16) с заменой $ u $ на $ u' $ получим:

\begin{equation}
\begin{array}{c}
\nonumber

\dfrac{1}{r}\Omega_{1r}^2u' = -xe^{-rx}u'(0) + \dfrac{1}{r}\Omega_{1r}u' - \dfrac{1}{r^2}\Omega_{1r}^2u''.

\end{array}
\end{equation}

Но $ \dfrac{1}{r}\Omega_{1r}u' $ и $ \dfrac{1}{r^2}\Omega_{1r}^2u'' $ есть $ O(\dfrac{1}{r}) $. Действительно,
\begin{equation}
\begin{array}{c}
\nonumber

\dfrac{1}{r}\vert\Omega_{1r}u'\vert = \biggl\vert \int\limits_0^x e^{-r(x-t)}u'(t)dt \biggr\vert \leq \Vert u' \Vert_{C[0,1]} \int\limits_0^x e^{-r(x-t)}dt = \\\\
= \dfrac{1}{r}(1 - e^{-rx})\Vert u' \Vert_{C[0,1]} \leq \dfrac{1}{r} \Vert u' \Vert_{C[0,1]}.

\end{array}
\end{equation}

И точно так же

\begin{equation}
\begin{array}{c}
\nonumber

\dfrac{1}{r^2} \biggl\vert \Omega_{1r}^2u'' \biggr\vert = \biggl\vert \int\limits_0^x e^{-r(x-t)}(x-t)u''(t)dt \biggr\vert \leq \dfrac{1}{r} \Vert u'' \Vert_{C[0,1]}.

\end{array}
\end{equation}

Для произвольного $ l $ из (1.16) получаем:

\begin{equation}
\begin{array}{c}

\dfrac{1}{r} \Omega_{1r}^lu' = \dfrac{1}{r} \varphi_{l-1}(r,x)u'(0) + \dfrac{1}{r}\Omega_{1r}^{l-1}u' - \dfrac{1}{r^2} \Omega_{1r}^lu''.

\end{array}
\end{equation}

Из (1.15) имеем:

\begin{equation}
\begin{array}{c}

\dfrac{1}{r}\Omega_{1r}^{l-1}u' = r^{l-2}\int\limits_0^x e^{-r(x-t)}\dfrac{(x-t)^{l-2}}{(l-2)!}u'(t)dt,

\end{array}
\end{equation}

\begin{equation}
\begin{array}{c}

\dfrac{1}{r^2} \Omega_{1r}^lu'' = r^{l-2}\int\limits_0^x e^{-r(x-t)}\dfrac{(x-t)^{l-1}}{(l-1)!}u''(t)dt.

\end{array}
\end{equation}

Берём интегралы в правых частях (1.19) и (1.20) по частям, каждый раз "перебрасывая" производную на функцию $ u'(t) $ до тех пор, пока перед интегралами не исчезнут степени $ r $, т.е. интегрируем $ l-2 $ раза. Тогда в (1.19) мы придём к интегралу $ \int\limits_0^x e^{-r(x-t)}u^{(l-1)}(t)dt $  а в (1.20) - к интегралу $ \int\limits_0^x e^{-r(x-t)}(x - t)u^l(t)dt $ которые имеют оценки $ O\biggl( \dfrac{1}{r}\Vert u^{(l - 1)} \Vert_{C[0,1]} \biggr) $ и $ O\biggl( \dfrac{1}{r}\Vert u^{(l)} \Vert_{C[0,1]} \biggr) $ соответственно.

Подстановки, которые получаются при интегрировании, будут состоять из функций $ \varphi_m(r,x), m = 1,...,l-3 $ - в формуле (1.19); $ m = 1,...,l-2 $ - в формуле (1.20), умноженных на значения производных функции $ u(x) $ в нуле до $ l-2 $ порядка включительно для формулы (1.19) и до $ l-1 $ порядка включительно для формулы (1.20). Общая сумма этих подстановок и первого слагаемого, стоящего в правой части выражения (1.18), будет иметь порядок $ r^{l-1}e^{-r\varepsilon} $ на отрезке $ [\varepsilon ,1] $.
Из вышесказанного следует, что соотношения (1.17) выполняются для любой $ u(x) \in C^k[0,1] $. Но множество функций, $ k $ раз непрерывно дифференцируемых, всюду плотно в пространстве $ C[0,1] $ по теореме Вейерштрасса~\eqref{theorem Weierstrass}.

Далее нормы операторов $ \Omega_{1r}^k $, рассматриваемых как операторы из $ C[0,1] $ в $ C_\varepsilon [0,1] $, ограничены константами, не зависящими от $ r $, поскольку из (1.8) следует:

\begin{equation}
\begin{array}{c}
\nonumber

\Vert \Omega_{1r}^ku \Vert_{C[\varepsilon ,1} = \Vert \Omega_{1r}(\Omega_{1r}^{k-1})u \Vert_{C[\varepsilon ,1]} = \Vert \Omega_{1r}\Omega_{1r}...(\Omega_{1r}u) \Vert_{C[\varepsilon ,1]} \leq \Vert u \Vert_{C[0,1]}.

\end{array}
\end{equation}


По теореме Бахана-Штейнгауcа~\eqref{theorem Banach Steinhaus} соотношение (1.17) справедливо для любой $ u \in C[0,1] $.

Выясним вопрос о приближении производных с помощью операторов $ \Omega_{1r}^k $. Обозначим $ D^m\Omega_{1r}^ku = \dfrac{d^m}{dx^m}\Omega_{1r}^ku, D^1 \equiv D $.

\section{Лемма 1.8. Формула производной от резольвенты \\ порядка $ m $}
\label{lemma1.8}

\textit{Операторы $ D^m\Omega_{1r}^ku $ при $ k \geq 1, m = 0,...,k-1 $ имеют вид}

\begin{equation}
\begin{array}{c}

D^m\Omega_{1r}^ku = r^k\int\limits_0^x K_{1m}(x,t,k,r)u(t)dt,

\end{array}
\end{equation}

\textit{где}
\begin{equation}
\begin{array}{c}

K_{1m}(x,t,k,r) = (-1)^me^{-r(x-t)} \biggl[r^m\dfrac{(x-t)^{k-1}}{(k-1)!} - mr^{m-1}\dfrac{(x-t)^{k-2}}{(k-2)!} + \\ + C_m^2r^{m-2}\dfrac{(x-t)^{k-3}}{(k-3)!} + ... + (-1)^{m-1}C_m^{m-1}r\dfrac{(x-t)^{k-m}}{(k-m)!} + \\ 
+ (-1)^m\dfrac{(x-t)^{k-m-1}}{(k-m-1)!}\biggr].

\end{array}
\end{equation}

\textbf{Доказательство.} По лемме 1.5~\eqref{lemma1.5} операторы $ \Omega_{1r}^k $ имеют вид:

\begin{equation}
\begin{array}{c}
\nonumber

\Omega_{1r}^ku = r^k \int\limits_0^x \dfrac{(x-t)^{k-1}}{(k-1)!}e^{-r(x-t)}u(t)dt, k = 1,2,...

\end{array}
\end{equation}

Если $ k \geq 2 $, а $ m = 1,2,...,k-2 $, то, очевидно, $ D^m(x-t)^{k-1} = 0 $ при $ t = x $. Эти производные присутствуют в выражениях для $ D^m\Omega_{1r}^ku $ при $ k \geq 2 $, а $ m = 1,2,...,k-1 $. Действительно, обозначим

\begin{equation}
\begin{array}{c}

K_{10}(x,t,k,r) = e^{-r(x-t)}\dfrac{(x-t)^{k-1}}{(k-1)!}.

\end{array}
\end{equation}
Тогда будем иметь:
\begin{equation}
\begin{array}{c}
\nonumber

\Omega_{1r}^ku = r^k \int\limits_0^x K_{10}(x,t,k,r)u(t)dt, \\
D\Omega_{1r}^ku = r^kK_{10}(x,t,k,r)_{|t=x} + r^k\int\limits_0^x DK_{10}(x,t,k,r)u(t)dt = \\
= r^k\int\limits_0^x DK_{10}(x,t,k,r)u(t)dt, \\
D^2\Omega_{1r}^ku = r^kK_{10}(x,t,k,r)_{|_{t=x}} + r^k\int\limits_0^x D^2K_{10}(x,t,k,r)u(t)dt.

\end{array}
\end{equation}

В выражении $ DK_{10}(x,t,k,r) $, очевидно, будут присутствовать степени $ (x-t)^l $, начиная с $ l = k - 2 $, поэтому, $ DK_{10}(x,t,k,r)_{|t=x} = 0 $ и

\begin{equation}
\begin{array}{c}
\nonumber

D^2\Omega_{1r}^ku = r^k\int\limits_0^x D^2K_{10}(x,t,k,r)u(t)dt.

\end{array}
\end{equation}


Продолжая процесс дифференцирования с учетом того, что $ D^2K_{10}(x,t,k,r) $ содержит степени $ (x-t)^l $, начиная с $ l = k - 3 $, $ D^3K_{10}(x,t,k,r) \rightarrow c_0 $  степени $ l = k - 4 $, $ D^{k-2}K_{10}(x,t,k,r)_{|t=x} $ - начиная со степени $ l = 1 $. Поскольку $ D^{k-2}K_{10}(x,t,k,r)_{|t=x} $ присутствует в выражении $ D^{k-1}\Omega_{1r}^ku $, то получаем, что для любого $ m \leq k - 1, k \geq 2 $ справедлива формула:

\begin{equation}
\begin{array}{c}
\nonumber

D^m\Omega_{1r}^ku = r^k\int\limits_0^x D^mK_{10}(x,t,k,r)u(t)dt

\end{array}
\end{equation}

Или, в соответствии с (1.23)

\begin{equation}
\begin{array}{c}

D^m\Omega_{1r}^ku = r^k\int\limits_0^x D^m\biggl[\dfrac{(x-t)^{k-1}}{(k-1)!}e^{-r(x-t)}\biggr]u(t)dt.

\end{array}
\end{equation}

Отсюда видно, что для указанных значений $ m $ и $ k $ производные $ D^m\Omega_{1r}^ku $ имеют интегральный вид.

При дальнейшем дифференцировании подстановки при $ t = x $ уже не будут равны нулю.


Найдем конкретные выражения для $ D^mK_{10}(x,t,k,r) $.


Обозначим для простоты $ \varphi_l(x,t) = \dfrac{(x-t)^l}{l!} $.
Тогда $ D^mK_{10}(x,t,k,r) = D^m[e^{-r(x-t)}\varphi_{k-1}(x,t)]$.
Учтем, что $ D\varphi_l(x,t) = \varphi_{l-1}(x,t) $.
Тогда получим
\begin{equation}
\begin{array}{c}
\nonumber

DK_{10}(x,t,k,r) = e^{-r(x-t)}[-r\varphi_{k-1}(x,t) + \varphi_{k-2}(x,t)] = \\
= -e^{-r(x-t)}[r\varphi_{k-1}(x,t) - \varphi_{k-2}(x,t)], \\
D^2K_{10}(x,t,k,r) = e^{-r(x-t)}\lbrace r^2\varphi_{k-1}(x,t) - r\varphi_{k-2}(x,t) - \\ - [r\varphi_{k-2}(x,t) - \varphi_{k-3}(x,t)]\rbrace = e^{-r(x-t)}[r^2\varphi_{k-1}(x,t) - \\ -2r\varphi_{k-2}(x,t) + \varphi_{k-3}(x,t)], \\
D^3K_{10}(x,t,k,r) = e^{-r(x-t)}[r^3\varphi_{k-1}(x,t) -  2r^2\varphi_{k-2}(x,t) + \\ + r\varphi_{k-3}(x,t)] + e^{-r(x-t)}[r^2\varphi_{k-2}(x,t) - 2r\varphi_{k-3}(x,t) + \varphi_{k-4}(x,t)] = \\
= -e^{-r(x-t)}\lbrace r^3\varphi_{k-1}(x,t) - 2r^2\varphi_{k-2}(x,t) + r\varphi_{k-3}(x,t) - \\ - r^2\varphi_{k-2}(x,t) + 2r\varphi_{k-3}(x,t) - \varphi_{k-4}(x,t)\rbrace = \\ = -e^{-r(x-t)}[r^3\varphi_{k-1}(x,t) -3r^2\varphi_{k-2}(x,t) + 3r\varphi_{k-3}(x,t) - \varphi_{k-4}(x,t)].

\end{array}
\end{equation}

Применяем метод математической индукции.

Пусть для $ m = l $ выполняется:

\begin{equation}
\begin{array}{c}

D^lK_{10}(x,t,k,r) = (-1)^le^{-r(x-t)}[r^l\varphi_{k-1}(x,t) - \\ - lr^{l-1}\varphi_{k-2}(x,t) + C_l^2r^{l-2}\varphi_{k-3}(x,t) + ... + \\ + (-1)^{l-1}C_l^{l-1}r\varphi_{k-l}(x,t) + (-1)^l\varphi_{k-l-1}(x,t)].

\end{array}
\end{equation}

Найдём $ D^{l+1}K_{10}(x,t,k,r) $. Имеем из (1.25):
\begin{equation}
\begin{array}{c}
\nonumber

D^{l+1}K_{10}(x,t,k,r) = D(D^lK_{10}(x,t,k,r)) = D\lbrace (-1)^le^{-r(x-t)}[r^l\varphi_{k-1}(x,t) - \\
- lr^{l-1}\varphi_{k-2}(x,t) + C_l^2r^{l-2}\varphi_{k-3}(x,t) + ... + \\ + (-1)^{l-1}C_l^{l-1}r\varphi_{k-l}(x,t) + (-1)^l\varphi_{k-l-1}(x,t)\rbrace .

\end{array}
\end{equation}
Получим:
\begin{equation}
\begin{array}{c}
\nonumber

D^{l+1}K_{10}(x,t,k,r) = (-1)^{l+1}e^{-r(x-t)}[r^{l+1}\varphi_{k-1}(x,t) - lr^l\varphi_{k-2}(x,t) + \\ + C_l^2r^{l-1}\varphi_{k-3}(x,t) + ... + (-1)^lC_l^{l-1}r^2\varphi_{k-1}(x,t) + (-1)^lr\varphi_{k-l-1}(x,t)] + \\ + (-1)^le^{r(x-t)}[r^l\varphi_{k-2}(x,t) - lr^{l-1}\varphi_{k-3}(x,t) + C_l^2r^{l-2}\varphi_{k-4}(x,t) + ... + \\ + (-1)^{l-1}C_l^{l-1}r\varphi_{k-l-1}(x,t) + (-1)^l\varphi_{k-l-2}(x,t)] = \\
= (-1)^{l+1}e^{-r(x-t)}\lbrace r^{l+1}\varphi_{k-1}(x,t) - lr^l\varphi_{k-2}(x,t) + C_l^2r^{l-1}\varphi_{k-3}(x,t) + \\ + ... + (-1)^{l-1}C_l^{l-1}r^2\varphi_{k-l}(x,t) + (-1)^lr\varphi_{k-l-1}(x,t) - r^l\varphi_{k-2}(x,t) + \\ + lr^{l-1}\varphi_{k-3}(x,t) - C_l^2r^{l-2}\varphi_{k-4}(x,t) + ... + (-1)^{l-1}C_l^{l-1}\varphi_{k-l-1}(x,t) - \\ - (-1)^l\varphi_{k-l-2}(x,t)\rbrace .

\end{array}
\end{equation}

Соберём члены с одинаковыми степенями $ r $. Тогда при $ r^l\varphi_{k-2}(x,t) $ с точностью до знака будет стоять коэффициент $ l + 1 $, при $ r^{l-1}\varphi_{k-3}(x,t) - C_l^2 + l = \frac{l(l-1)}{2} + l = \frac{(l+1)l}{2} = C_{l+1}^2$.
В общем случае при $ r^{l-j}\varphi_{k-2-j} $ будет коэффициент

\begin{equation}
\begin{array}{c}
\nonumber

C_l^j + C_l^{j+1} = \dfrac{l(l-1)...(l-j+1)}{j!} + \dfrac{l(l-1)...(l-j)}{(j+1)!} = \\
= \dfrac{l(l-1)(l-j+1)}{(j+1)!}(j+1+l-j) = C_{l+1}^{j+1}.

\end{array}
\end{equation}

Отсюда получаем формулу (1.25) с заменой $ l $ на $ l + 1 $. Наконец, подставляя в $ m $ вместо $ l $ выражения $ \varphi_l(x,t) $, получим утверждение леммы 1.8~\eqref{lemma1.8}.

\section{Лемма 1.9. Приближающие свойства производной от резольвенты порядка $ m $ в пространствах гладких функций.}
\label{lemma1.9}
\textit{При $ k \geq 2, m=1,...,k-1 $ для любой функции $ u(x) \in C^{k-1}[0,1] $ справедливы соотношения:}

\begin{equation}
\begin{array}{c}

\Vert D^m\Omega_{1r}^ku - u^{(m)} \Vert_{C[\varepsilon ,1]} \rightarrow 0 $ \textit{при} $ r \rightarrow \infty,

\end{array}
\end{equation}
\textit{Где $ D^m\Omega_{1r}^ku $ определены в (1.21)-(1.22).}

\textbf{Доказательство.} Пусть $ k = 2 $. Тогда в соответствии с (1.19), интегрируя по частям, получим:
\begin{equation}
\begin{array}{c}
\nonumber

D\Omega_{1r}^2u = r^2\int\limits_0^x \dfrac{d}{dx}[e^{-r(x-t)}(x-t)]u(t)dt = \\ = -r^2\int\limits_0^x \dfrac{d}{dx}[e^{-r(x-t)}(x-t)]u(t)dt = r^2xe^{-rx}u(0) + \Omega_{1r}^2u'.

\end{array}
\end{equation}
Отсюда получаем:
\begin{equation}
\begin{array}{c}
\nonumber

\Vert D\Omega_{1r}^2u - u' \Vert_{C[\varepsilon ,1]} \leq r^2e^{-r\varepsilon}\vert u(0) \vert + \Vert \Omega_{1r}^2u' - u' \Vert_{C[\varepsilon ,1]} \rightarrow 0 $ при $ r \rightarrow \infty

\end{array}
\end{equation}
по лемме 1.7~\eqref{lemma1.7}.

Для $ k = 3 $ имеем:
\begin{equation}
\begin{array}{c}

D\Omega_{1r}^3u = r^3 \int\limits_0^x \dfrac{d}{dx} \biggl[ e^{-r(x-t)}\dfrac{(x-t)^2}{2} \biggr] u(t)dt = \\
= -r^3 \int\limits_0^x \dfrac{d}{dx} \biggl[ e^{-r(x-t)}\dfrac{(x-t)^2}{2} \biggr] u(t)dt =
r^3\dfrac{x^2}{2}e^{-rx}u(0) + \Omega_{1r}^3u',

\end{array}
\end{equation}
И точно так же, как для $ k = 2 $,
\begin{equation}
\begin{array}{c}

\Vert D\Omega_{1r}^3u - u' \Vert_{C[\varepsilon ,1]} \leq r^3e^{-r\varepsilon}\vert u(0) \vert + \Vert \Omega_{1r}^3u' - u' \Vert_{C[\varepsilon ,1]} \rightarrow 0 $ при $ r \rightarrow \infty

\end{array}
\end{equation}

Далее, из (27) получаем:

\begin{equation}
\begin{array}{c}

D^2\Omega_{1r}^3u = D(D\Omega_{1r}^3u) = r^3D(\dfrac{x^2}{2}e^{-rx})u(0) + D\Omega_{1r}^3u'.

\end{array}
\end{equation}
Снова применяем формулу (27) с заменой  на  и получаем:
\begin{equation}
\begin{array}{c}
\nonumber

\Vert D^2\Omega_{1r}^3u - u'' \Vert_{C[\varepsilon ,1]} \leq \dfrac{3}{2}r^4e^{-r\varepsilon}(\vert u(0)\vert + \vert u'(0) \vert) + \Vert D\Omega_{1r}^3u' - u'' \Vert_{C[\varepsilon ,1]}.

\end{array}
\end{equation}
Из (28) заменяя $ u $ на $ u' $, получаем, что $ \Vert D^2\Omega_{1r}^3u - u'' \Vert_{C[\varepsilon ,1]} \rightarrow 0 $ при $ r \rightarrow \infty $.

Действуем так же, как и в случае любого $ m $, т.е. сначала получаем формулу, аналогичную (1.29), а затем пользуемся леммой 1.7~\eqref{lemma1.7}.

В соответствии с (1.24) имеем:
\begin{equation}
\begin{array}{c}
\nonumber

D^m\Omega_{1r}^ku = r^k\int\limits_0^x D^m \biggl[ \dfrac{(x-t)^{k-1}}{(k-1)!}e^{-r(x-t)} \biggr]u(t)dt = r^k\int\limits_0^x D^mK_{10}(x,t,k,r)u(t)dt.

\end{array}
\end{equation}
Далее,
\begin{equation}
\begin{array}{c}

\int\limits_0^x D^mK_{10}(x,t,k,r)u(t)dt = \int\limits_0^x D(D^{m-1}K_{10}(x,t,k,r))u(t)dt = \\ 
= - \int\limits_0^x \dfrac{d}{dt}(D^{m-1}K_{10}(x,t,k,r))u(t)dt = [D^{m-1}K_{10}(x,t,k,r)]_{t=0}u(0) + \\ \int\limits_0^x D^{m-1}K_{10}(x,t,k,r)u'(t)dt = [D^{m-1}K_{10}(x,t,k,r)]_{t=0}u(0) + \\ + [D^{m-2}K_{10}(x,t,k,r)]_{t=0}u'(0) + ... + [D^2K_{10}(x,t,k,r)]_{t=0}u^{(m-3)}(0) + \\ + [DK_{10}(x,t,k,r)]_{t=0}u^{(m-2)}(0) + K_{10}(x,t,k,r)_{t=x}u^{(m-1)}(0) + \\ + \int\limits_0^x K_{10}(x,t,k,r)u^{(m)}(t)dt.

\end{array}
\end{equation}
Поскольку в лемме 1.8~\eqref{lemma1.8} $ D^lK_{10}(x,t,k,r) = K_{1l}(x,t,k,r) $, где $ K_{1l}(x,t,k,r) $ имеет вид (1.22) с заменой $ m $ на $ l $, то отсюда следует, что все подстановки в выражении (1.30) есть $ O(r^{m-1}e^{-r\varepsilon}) $ на отрезке $ [\varepsilon ,1] $.

Отсюда получаем:
\begin{equation}
\begin{array}{c}
\nonumber

\Vert D^m\Omega_{1r}^ku - u^{(m)} \Vert_{C[\varepsilon ,1]} = \Vert \Omega_{1r}^ku^{(m)} - u^{(m)} \Vert_{C[\varepsilon ,1]} + O(r^{k+m-1}e^{-r\varepsilon}).

\end{array}
\end{equation}
Из леммы 1.7~\eqref{lemma1.7} следует соотношение (1.26).

\chapter{Приближающие свойства резольвенты оператора \\ $ L_2:y', y(1)=0 $ на отрезке $ [0, 1 - \varepsilon] $.}
Рассмотрим оператор $ L_2: y', y(1) = 0 $, отличающийся от оператора $ L_1 $ лишь граничным условием.

Обозначим его резольвенту $R_\lambda(L_2)$. Положим $ \lambda = r, r > 0 $ и рассмотрим оператор $ -rR_r(L_2) $.

Получим аналоги лемм 1.1-1.9.

\section{Лемма 2.1. Формула резольвенты дифференциального оператор первого порядка.}
\label{lemma2.1} 
\textit{Для $ y(x) = R_\lambda(L_2) $ имеет место формула:}
\begin{equation}
\begin{array}{c}

y(x) \equiv R_\lambda(L_2)u = -\int\limits_x^1 e^{\lambda (x-t)}u(t)dt.

\end{array}
\end{equation}
\textbf{Доказательство.} Если $ y = R_\lambda(L_2) $, то

\begin{equation}
\begin{array}{c}

y' - \lambda y = u,

\end{array}
\end{equation}
\begin{equation}
\begin{array}{c}

y(1) = 0.

\end{array}
\end{equation}

Общее решение уравнения (2.2) из доказательства леммы 1.1~\eqref{lemma1.1} имеет вид (1.4).

Найдём $ С $ из условия (2.3):
\begin{equation}
\begin{array}{c}
\nonumber

Ce^\lambda + \int\limits_0^1 e^{\lambda (1-t)}u(t)dt = 0,

\end{array}
\end{equation}
откуда
\begin{equation}
\begin{array}{c}

C = - \int\limits_0^1 e^{-\lambda t}u(t)dt.

\end{array}
\end{equation}
Подставив (2.4) в (1.4), получим:
\begin{equation}
\begin{array}{c}
\nonumber

y(x) - -e^{\lambda t}\int\limits_0^1 e^{-\lambda t}u(t)dt + \int\limits_0^x e^{\lambda (x-t)}u(t)dt = -\int\limits_x^1 e^{\lambda (x-t)}u(t)dt,

\end{array}
\end{equation}
что и требовалось доказать.

Положим в (2.1) $ \lambda =-r $, где $ r > 0 $ и рассмотрим операторы $ -rR_r(L_2) $.

\section{Лемма 2.2. Приближающие свойства резольвенты дифференциального оператор первого порядка.}
\label{lemma2.2}
\textit{Для любой непрерывной функции $ u(x) $ имеет место сходимость:}
\begin{equation}
\begin{array}{c}

\Vert -rR_r(L_2)u - u \Vert_{C[0,1-\varepsilon]} \rightarrow 0 $ \textit{при} $ r \rightarrow \infty

\end{array}
\end{equation}
\textit{где $ \varepsilon $ - любое малое положительное число.}

\textbf{Доказательство.} Так же, как в лемме 1.2~\eqref{lemma1.2}, сначала докажем сходимость (2.5) для $ u(x) \in C^1[0,1] $. Тогда имеем:
\begin{equation}
\begin{array}{c}
\nonumber

\int\limits_x^1 e^{r(x-t)}u(t)dt = e^{rx}\int\limits_x^1 e^{-rt}u(t)dt = \\
= -e^{rx}\dfrac{1}{r}\biggl[ e^{-rt}u(t)dt\bigg\vert_x^1 - \int\limits_x^1 e^{-rx}u'(t)dt\biggr] = \\ = \dfrac{1}{r}u(x) - \dfrac{1}{r}e^{-r(1-x)}u(1) + \dfrac{1}{r}\int\limits_x^1 e^{r(x-t)}u'(t)dt.

\end{array}
\end{equation}
Отсюда имеем:
\begin{equation}
\begin{array}{c}

\Vert -rR_r(L_2)u - u \Vert_{C[0,1-\varepsilon]} \leq \\ \leq \vert u(1) \vert e^{-r\varepsilon} + \Vert u' \Vert_{C[0,1]}\times\biggl\Vert \int\limits_x^1 e^{r(x-t)}dt\biggr\Vert_{C[0,1-\varepsilon]},

\end{array}
\end{equation}
\begin{equation}
\begin{array}{c}

\int\limits_x^1 e^{r(x-t)}dt = -\dfrac{1}{r}(e^{-r(1-x)}-1) \leq \dfrac{1}{r},

\end{array}
\end{equation}
а из этой оценки и оценки (2.6) получаем (2.5).

Пусть теперь $ u(x) \in C[0,1] $. Покажем, что нормы операторов $ -rR_r(L_2) $, рассматриваемых как операторы из $ C[0,1] $ в $ C[0,1-\varepsilon] $, ограничены константой, не зависящей от $ r $.

Действительно
\begin{equation}
\begin{array}{c}
\nonumber

\Vert -rR_r(L_2)u \Vert_{C[0,1-\varepsilon]} \leq \Vert -rR_r(L_2)u \Vert_{C[0,1]} = \biggl\Vert r\int\limits_x^1 e^{r(x-t)}u(t)dt\biggr\Vert_{C[0,1]} \leq \Vert u \Vert_{C[0,1]}

\end{array}
\end{equation}
в силу оценки (2.7).

Далее, как в лемме 1.2~\eqref{lemma1.2}, применяем теорему Банаха-Штейнгауcа~\eqref{theorem Banach Steinhaus} к операторам $ -rR_r(L_2) $ и получаем утверждение леммы 2.2~\eqref{lemma2.2}.

Теперь займёмся приближающими свойствами операторов $ -rR_r(L_2) $ в пространстве $ C^l[0,1] $.

Пусть сначала $ u(x) \in C^{l-1}[0,1] $. Рассмотрим операторы

\begin{equation}
\begin{array}{c}
\nonumber

D^kR_r(L_2)u \equiv (R_r(L_2)u)_x^{(k)}, k = 1,...,l, D^1 \equiv D (Du = u').

\end{array}
\end{equation}

\section{Лемма 2.3. Формула резольвенты в пространствах гладких функций}
\label{lemma2.3}
\textit{Операторы $ D^kR_r(L_2) $ имеют вид:}
\begin{equation}
\begin{array}{c}

D^kR_r(L_2)u = u^{(k-1)}(x) - ru^{(k-2)}(x) + r^2u^{(k-3)}(x) + ... + \\
+ (-1)^{k-1}r^{k-1}u(x) + (-1)^kr^k\int\limits_x^1 e^{r(x-t)}u(t)dt.

\end{array}
\end{equation}

\textbf{Доказательство.} Для $ k = 1 $ имеем:
\begin{equation}
\begin{array}{c}
\nonumber

DR_r(L_2)u = (R_r(L_2)u)_x' = \biggl( -\int\limits_x^1 e^(r(x-t))u(t)dt \biggr)_x' = u(x) - r\int\limits_x^1 e^{r(x-t)}u(t)dt.

\end{array}
\end{equation}

Применяем метод математической индукции, как и в доказательстве леммы 1.3~\eqref{lemma1.3}. Все выкладки повторяются с заменой интеграла $ \int\limits_0^x $ на интеграл $ \int\limits_x^1 $, а экспоненты $ e^{-r(x-t)} $ на экспоненту $ e^{r(x-t)} $. В результате получаем формулу (2.8).

\section{Лемма 2.4. Приближающие свойства резольвенты в пространствах гладких функций.}
\label{lemma2.4}
\textit{Если $ u(x) \in C^l[0,1] $, то имеет место сходимость:}
\begin{equation}
\begin{array}{c}

\Vert -rD^kR(L_2)u - u^{(k)}(x) \Vert_{C[0,1-\varepsilon]} \rightarrow 0 $ \textit{при} $ r \rightarrow \infty, k = 1,...,l.

\end{array}
\end{equation}

\textbf{Доказательство.} Пусть $ k = 1 $. По лемме 2.3~\eqref{lemma2.3} имеем:
\begin{equation}
\begin{array}{c}

DR_r(L_2)u = u(x) - r\int_x^1 e^{e(x-t)}u(t)dt.

\end{array}
\end{equation}

Далее
\begin{equation}
\begin{array}{c}

\int\limits_x^1 e^{r(x-t)}u(t)dt = e^{rx}\int\limits_x^1 e^{-rx}u(t)dt = -\dfrac{1}{r}e^{rx}(e^{-rx}u(t))\bigg|_x^1 + \dfrac{1}{r}\int\limits_x^1 e^{r(x-t)}u'(t)dt = \\ = \dfrac{1}{r}u(x) - \dfrac{1}{r}e^{-r(1-x)}u(1) + \dfrac{1}{r}\int\limits_x^1 e^{r(x-t)}u'(t)dt.

\end{array}
\end{equation}

Подставляя (2.11) в (2.10), получим:
\begin{equation}
\begin{array}{c}
\nonumber

DR_r(L_2)u = e^{-r(1-x)}u(1) - \int\limits_x^1 e^{r(x-t)}u'(t)dt,

\end{array}
\end{equation}
или
\begin{equation}
\begin{array}{c}
\nonumber

DR_r(L_2)u = R_r(L_2)u' + e^{-r(1-x)}u(1).

\end{array}
\end{equation}

Повторяем рассуждения, рпиведённые в лемме 1.4~\eqref{lemma1.4} с заменой $ R_{-r}(L_1) $ на $ R_r(L_2) $, $ e^{-rx} $ на $ e^{-r(1-x)} $, интеграла $ \int\limits_0^x $ на интеграл $ \int\limits_x^1 $.

Тогда приходим к формуле:
\begin{equation}
\begin{array}{c}

D^kR_r(L_2)u = R_r(L_2)u^{(k)} + e^{-r(1-x)}u^{(k-1)}(1) - re^{r(1-x)}u^{(k-2)}(1) + ... + \\ + (-1)^{k-1}r^{k-1}e^{-r(1-x)}u(1).

\end{array}
\end{equation}

Из формулы (2.12) получаем оценку:
\begin{equation}
\begin{array}{c}
\nonumber

\Vert -rD^kR_r(L_2)u - u^{(k)} \Vert_{C[0,1-\varepsilon]} \leq \Vert -rR_r(L_2)u^{(k)} - u^{(k)} \Vert_{C[0,1-\varepsilon]} + \\ + e^{-r\varepsilon}\sum\limits_{j=0}^{k-1} r^j\vert u^{k-j-1}(1)\vert ,

\end{array}
\end{equation}
и тогда сходимость (2.9) вытекает из леммы 2.2~\eqref{lemma2.2}.

\textbf{Замечание.} \textit{Если в лемме 2.2~\eqref{lemma2.2} $ u(1) = 0 $, то сходимость (2.5) будет выполняться при $ \varepsilon = 0 $, т.е. на всём отрезке $ [0,1] $. Если в лемме 2.4~\eqref{lemma2.4} $ u^{(m)}(1) = 0, m = 0,1,...,k $, то сходимость (2.9) будет выполняться также при $ \varepsilon = 0 $.}

Обозначим $ -rR_r(L_2) = \Omega_{2r} $ и рассмотрим свойства степеней $ \Omega_{2r}^k $.

\section{Лемма 2.5. Формула степеней резольвенты}
\label{lemma2.5}
\textit{операторы $ \Omega_{2r}^k $ имеют вид:}
\begin{equation}
\begin{array}{c}

\Omega_{2r}^ku = r^k\int\limits_x^1 \dfrac{(t-x)^{k-1}}{(k-1)!}e^{r(x-t)}u(t)dt.

\end{array}
\end{equation}

\textbf{Доказательство.} Для $ k = 2 $ имеем:

\begin{equation}
\begin{array}{c}
\nonumber

\Omega_{2r}^2u = r^2\int\limits_x^1 e^{r(x-t)}dt\int\limits_t^1 e^{r(t-\tau)}u(\tau)d\tau = r^2e^{rx} \int\limits_x^1 dt \int\limits_x^1 \varepsilon (\tau ,t)e^{-r\tau}u(\tau)d\tau ,

\end{array}
\end{equation}
где $ \varepsilon (\tau ,t) = 1 $ при $ t \leq \tau $ и $ \varepsilon (\tau ,t) = 0 $ при $ t \geq \tau $.

Меняем порядок интегрирования:
\begin{equation}
\begin{array}{c}
\nonumber

\Omega_{2r}^2u = r^2e^{rx} \int\limits_x^1 e^{-r\tau}u(\tau)d\tau \int\limits_x^1 \varepsilon (\tau ,t) dt = r^2e^{rx} \int\limits_x^1 e^{-r\tau}u(\tau)d\tau \biggl[ \int\limits_x^{\tau} \varepsilon (\tau ,t)dt + \\ + \int\limits_{\tau}^1 \varepsilon (\tau ,t)dt \biggr] = r^2e^{rx} \int\limits_x^1 e^{-r\tau}u(\tau)d\tau \int\limits_x^{\tau}dt = r^2 \int\limits_x^1(\tau - x)e^{r(x-\tau)}u(\tau)d\tau .

\end{array}
\end{equation}
Меняем обозначение $ \tau $ на $ t $ , получаем:
\begin{equation}
\begin{array}{c}
\nonumber

\Omega_{2r}^2u = r^2\int\limits_x^1 (t-x)e^{r(x-t)}u(t)dt.

\end{array}
\end{equation}

Повторяем рассуждения, приведённые в доказательстве леммы 1.5~\eqref{lemma1.5}, приходим к утверждению леммы 2.5~\eqref{lemma2.5}.

\section{Лемма 2.6. Формула степеней резольвенты в пространстве непрерывно дифференцируемых функций первого порядка}
\label{lemma2.6}
\textit{Если $ u(x) \in C^1[0,1] $, то операторы $ \Omega_{2r}^k $ имеют представление:}
\begin{equation}
\begin{array}{c}

\Omega_{2r}^ku = -\dfrac{r^{k-1}(1-x)^{k-1}e^{-r(1-x)}}{(k-1)!}u(1) + \Omega_{2r}^{k-1}u + \dfrac{1}{r}\Omega_{2r}^ku'.

\end{array}
\end{equation}

\textbf{Доказательство.} Пусть $ k = 2 $. Тогда из (2.13) получаем:
\begin{equation}
\begin{array}{c}
\nonumber

\Omega_{2r}^2u = r^2\int\limits_x^1 (t-x)e^{r(x-t)}u(t)dt.

\end{array}
\end{equation}

Интегрируем по частям, получаем:
\begin{equation}
\begin{array}{c}
\nonumber

\Omega_{2r}^2u = r^2\biggl\lbrace -\dfrac{1}{r}[(t-x)e^{r(x-t)}u(t)]_x^1 + \dfrac{1}{r}\int\limits_x^1 e^{r(x-t)}[(t-x)u(t)]_t'dt\biggr\rbrace = \\ = -r(1-x)e^{-r(1-x)}u(1) + r\int\limits_x^1 e^{r(x-t)}[u(t) + (t-x)u'(t)]_t'dt = \\ = -r(1-x)e^{-r(1-x)}u(1) + r \int\limits_x^1 e^{r(x-t)}u(t)dt + \\ + r \int\limits_x^1 (t-x)e^{r(x-t)}u'(t)dt = -r(1-x)e^{-r(1-x)}u(1) + \Omega_{2r}u + \dfrac{1}{r}\Omega_{2r}u'.

\end{array}
\end{equation}

Применяем метод математической индукции, как и в лемме 1.6~\eqref{lemma1.6}, приходим к утверждению леммы 2.6~\eqref{lemma2.6}.

\section{Лемма 2.7. Приближающие свойства резольвенты в пространствах непрерывных функций.}
\label{lemma2.7}

\textit{Для $ u(x) \ in C[0,1] $ справедливы соотношения:}
\begin{equation}
\begin{array}{c}

\Vert \Omega_{2r}^ku - u \Vert_{C[0,1-\varepsilon]} \rightarrow 0 $ \textit{при} $ r \rightarrow \infty, k = 1,2,...

\end{array}
\end{equation}

\textbf{Доказательство.} Для $ k = 1 $ соотношение (2.15) доказано в лемме 2.2~\eqref{lemma2.2}. Пусть $ k \geq 2 $, а $ u(x) \in C^k[0,1] $. 

Обозначим $ \widetilde\varphi_l(r,x) = - \dfrac{r^l(1-x)^le^{-r(1-x)}}{l!} $.

Из (2.15) имеем:
\begin{equation}
\begin{array}{c}
\nonumber

\Vert \Omega_{2r}^ku - u \Vert_{C[0,1\varepsilon]} \leq \Vert \widetilde\varphi_{k-1}(r,x)u(1)\Vert_{C[0,1-\varepsilon]} + \Vert \Omega_{2r}^{k-1}u - u \Vert_{C[0,1-\varepsilon]} + \\\\ + \biggl\Vert \dfrac{1}{r}\Omega_{2r}^ku'\biggr\Vert_{C[0,1-\varepsilon]} \leq \widetilde\varphi_{k-1}(r,x)u(1)\Vert_{C[0,1-\varepsilon]} + \widetilde\varphi_{k-2}(r,x)u(1)\Vert_{C[0,1-\varepsilon]} + \\ + ... + \widetilde\varphi_1(r,x)u(1)\Vert_{C[0,1-\varepsilon]} + \Vert \Omega_{2r}u - u \Vert_{C[0,1-\varepsilon]} + \biggl\Vert \dfrac{1}{r}\Omega_{2r}^ku'\biggr\Vert_{C[0,1-\varepsilon]} + \\ + \biggl\Vert \dfrac{1}{r}\Omega_{2r}^{k-1}u'\biggr\Vert_{C[0,1-\varepsilon]} + ... + \biggl\Vert \dfrac{1}{r}\Omega_{1r}^2u'\biggr\Vert_{C[0,1-\varepsilon]}.

\end{array}
\end{equation}

Далее, по аналогии с доказательством леммы 1.7~\eqref{lemma1.7}, имеем: $ \widetilde\varphi_l(r,x) \leq r^le^{-r\varepsilon} $ на отрезке $ [0,1-\varepsilon] $, по лемме 2.2~\eqref{lemma2.2} $ \Vert \Omega_{2r}u - u \Vert_{C[0,1-\varepsilon]} \rightarrow 0 $ при $ r \rightarrow \infty $ для любой $ u \in C[0,1] $. Осталось позазать, что $ \biggl\Vert \dfrac{1}{r}\Omega_2^lu' \biggr\Vert_{C[0,1-\varepsilon]} \rightarrow 0 $ при $ r \rightarrow \infty $ для $ l = 2,...,k $.

Пользуемся для этого формулой (2.14), применяя её к производным от функции $ u $.

Получим:
\begin{equation}
\begin{array}{c}
\nonumber

\dfrac{1}{r}\Omega_{2r}^lu' = \dfrac{1}{r} \widetilde\varphi_{l-1}(r,x)u'(1) + \dfrac{1}{r}\Omega_{2r}^{l-1}u' - \dfrac{1}{r^2}\Omega_{2r}^2u'',

\end{array}
\end{equation}
\begin{equation}
\begin{array}{c}

\dfrac{1}{r}\Omega_{2r}^{l-1}u' = r^{l-2}\int\limits_x^1 e^{r(x-t)}\dfrac{(t-x)^{l-2}}{(l-2)!}u'(t)dt,

\end{array}
\end{equation}
\begin{equation}
\begin{array}{c}

\dfrac{1}{r}\Omega_{2r}^lu'' = r^{l-2}\int\limits_x^1 e^{r(x-t)}\dfrac{(t-x)^{l-1}}{(l-1)!}u''(t)dt,

\end{array}
\end{equation}

Затем берём интегралы в правых частях (2.16)-(2.17) по частям $ l - 2 $ раза, каждый раз "перебрасывая" производную на функцию $ u'(t) $ в (2.16) и функцию $ u''(t) $ в (2.17) до тех пор, пока не исчезнут степени $ r $ перед интегралами.

Тогда в (2.16) мы придём к интегралу $ \int\limits_x^1 e^{r(x-t)}u^{(l-1)}(t)dt $, а в (2.17) - к интегралу $ \int\limits_x^1 e^{r(x-t)}u^{(l)}(t)dt $.

Эти интегралы, а также подстановки, полученные при интегрировании по частям, будут иметь те же оценки, что и аналогичные им интегралы и подстановки в доказательстве леммы 1.7~\eqref{lemma1.7}.

Отсюда получаем сходимость (2.15) для любой функции $ u \in C^k[0,1] $.

Далее доказываем ограниченность норм $ \Vert \Omega_{2r}^ku \Vert_{C[0,1] \rightarrow C[0,1-\varepsilon]} $:
\begin{equation}
\begin{array}{c}
\nonumber

\Vert \Omega_{2r}^ku \Vert_{C[0,1] \rightarrow C[0,1-\varepsilon]} = \Vert \Omega_{2r}(\Omega_{2r}^{k-1}u) \Vert_{C[0,1-\varepsilon]} = \\ = \Vert \Omega_{2r}\Omega_{2r}...(\Omega_{2r}u)\Vert_{C[0,1-\varepsilon]} \leq \Vert u \Vert_{C[0,1]}

\end{array}
\end{equation}

Наконец, пользуемся теоремой Банаха-Штейнгауcа~\eqref{theorem Banach Steinhaus} и приходим к утверждению леммы 2.7~\eqref{lemma2.7}.

Рассмотрим теперь операторы $ D^m\Omega_{2r}^ku = \dfrac{d^m}{dx^m}\Omega_{2r}^ku, D' = D $.

\section{Лемма 2.8. Формула производной от резольвенты \\ порядка $ m $}
\label{lemma2.8}

\textit{Операторы  имеют вид}
\begin{equation}
\begin{array}{c}

D^m\Omega_{2r}^ku = r^k\int\limits_x^1 K_{2m}(x,t,k,r)u(t)dt,

\end{array}
\end{equation}
\textit{где}
\begin{equation}
\begin{array}{c}

K_{2m}(x,t,k,r) = e^{r(x-t)} \biggl[ r^m\dfrac{(t-x)^{k-1}}{(k-1)!} - mr^{m-1}\dfrac{(t-x)^{k-2}}{(k-2)!} + \\ + C_m^2r^{m-2}\dfrac{(t-x)^{k-3}}{(k-3)!} + ... + (-1)^{m-1}C_m^{m-1}r\dfrac{(t-x)^{k-m}}{(k-m)!} + \\ + (-1)^m\dfrac{(t-x)^{k-m-1}}{(k-m-1)!}\biggr].

\end{array}
\end{equation}

\textbf{Доказательство.} Рассуждаем по аналогии с доказательством леммы 1.8~\eqref{lemma1.8}. обозначим $ K_{20}(x,t,k,r) = e^{r(x-t)}\dfrac{(t-x)^{k-1}}{(k-1)!} $, записываем представление (2.13) из леммы 2.5~\eqref{lemma2.5} в виде:
\begin{equation}
\begin{array}{c}
\nonumber

\Omega_{2r}^ku = r^k\int\limits_x^1 K_{20}(x,t,k,r)u(t)dt,

\end{array}
\end{equation}
учитываем, что в выражении для $ D^jK_{20}(x,t,k,r) $ при $ j=1,...,k-2 $ будут присутствовать степени $ t-x $, и тогда будет справедливо выражение:
\begin{equation}
\begin{array}{c}

D^m\Omega_{2r}^ku = r^k\int\limits_x^1 D^mK_{20}(x,t,k,r)dt = \\ = r^k\int\limits_x^1 D^m\biggl[\dfrac{(t-x)^{k-1}}{(k-1)!}e^{r(x-t)}\biggr]u(t)dt.

\end{array}
\end{equation}

Осталось найти конкретное выражение для $ D^mK_{20}(x,t,k,r) $.

Обозначим
\begin{equation}
\begin{array}{c}
\nonumber

\widetilde\varphi_l(x,t) = \dfrac{(t-x)^l}{l!}.

\end{array}
\end{equation}
Учтём, что $ D\widetilde\varphi_l(x,t) = -D\widetilde\varphi_{k-2}(x,t) $.

Тогда получим:
\begin{equation}
\begin{array}{c}
\nonumber

DK_{20}(x,t,k,r) = e^{r(x-t)}[-r\widetilde\varphi_{k-1}(x,t) - \widetilde\varphi_{k-2}(x,t)], \\\\
D^2K_{20}(x,t,k,r) = e^{r(x-t)}\lbrace r^2\widetilde\varphi_{k-1}(x,t)-r\widetilde\varphi_{k-2}(x,t) + \\ + D[r\widetilde\varphi_{k-1}(x,t) - \widetilde\varphi_{k-2}(x,t)]\rbrace = e^{r(x-t)}\lbrace r^2\widetilde\varphi_{k-1}(x,t) - \\ - r\widetilde\varphi_{k-2}(x,t) - r\widetilde\varphi_{k-2}(x,t) + \widetilde\varphi_{k-3}(x,t)\rbrace = \\ = e^{r(x-t)}[r^2\widetilde\varphi_{k-1}(x,t) - 2r\widetilde\varphi_{k-2}(x,t) + \widetilde\varphi_{k-3}(x,t)], \\\\
D^3K_{20}(x,t,k,r) = e^{r(x-t)}\lbrace r^3\widetilde\varphi_{k-1}(x,t)-2r^2\widetilde\varphi_{k-2}(x,t) + \\ + r\widetilde\varphi_{k-3}(x,t) + D[r^2\widetilde\varphi_{k-1}(x,t) - 2r\widetilde\varphi_{k-2}(x,t) + \widetilde\varphi_{k-3}(x,t)]\rbrace = \\ = e^{r(x-t)}\lbrace r^3\widetilde\varphi_{k-1}(x,t)-2r^2\widetilde\varphi_{k-2}(x,t) + r\widetilde\varphi_{k-3}(x,t) - r^2\widetilde\varphi_{k-2}(x,t) + \\ + 2r\widetilde\varphi_{k-3}(x,t) - \widetilde\varphi_{k-4}(x,t)\rbrace = e^{r(x-t)}[r^3\widetilde\varphi_{k-1}(x,t) - \\ - 3r^2\widetilde\varphi_{k-2}(x,t) + 3r\widetilde\varphi_{k-3}(x,t) - \widetilde\varphi_{k-4}(x,t)].

\end{array}
\end{equation}

Сравнивая полученные выражения с соответствующими выражениями в лемме 1.8~\eqref{lemma1.8}, видим, что они отличаются заменой $ e^{-r(x-t)} $ на $ e^{r(x-t)} $, $ \varphi_l(x,t) $ на $ \widetilde\varphi_l(x,t) $, а также заменой знака на противоположный при вычислении $ D^lK_{20}(x,t,k,r) $ при $ l $ - нечётном.

С учётом этого повторяем дословно дальнейшие рассуждения в доказательстве леммы 1.8~\eqref{lemma1.8} и приходим к утверждению леммы 2.8~\eqref{lemma2.8}.

\section{Лемма 2.9. Приближающие свойства производной от резольвенты порядка $ m $ в пространствах гладких функций.}
\label{lemma2.9}

\textit{При $ k \geq 2, m = 1,...,k-1 $ для любой функции $ u(x) \in C^{k-1}[0,1] $ справедливы соотношения:}
\begin{equation}
\begin{array}{c}

\Vert D^m\Omega_{2r}^ku - u^{(m)} \Vert_{C[0,1-\varepsilon]} \rightarrow 0 $ \textit{при} $ r \rightarrow \infty .

\end{array}
\end{equation}
\textbf{Доказательство.} Пусть $ k = 2 $. Тогда в соответствии с (2.20), интегрируя по частям, получим:
\begin{equation}
\begin{array}{c}
\nonumber

D\Omega_{2r}^2u = r^2 \int\limits_x^1 \dfrac{d}{dx}[e^{r(x-t)}(t-x)]u(t)dt = \\ = -r^2\int\limits_x^1 \dfrac{d}{dt}[e^{r(x-t)}(x-t)](u(t)dt = -r^2[e^{r(x-t)}(x-t)(u(t)]_x^1 + \\ + r^2\int\limits_x^1 e^{r(x-t)}(t-x)u'(t)dt = -r^2[e^{-r(1-x)}(1-x)(u(1) + r^2\int\limits_x^1 e^{r(x-t)}(t-x)u'(t)dt.

\end{array}
\end{equation}

Отсюда получаем:
\begin{equation}
\begin{array}{c}
\nonumber
\Vert D\Omega_{2r}^2u - u' \Vert_{C[0,1-\varepsilon]} \leq r^2e^{-r\varepsilon}\vert u(1) \vert + \Vert \Omega_{2r}^2u' - u' \Vert_{C[0,1-\varepsilon]} \rightarrow 0 $ при $ r \rightarrow \infty
\end{array}
\end{equation}
по лемме 2.7~\eqref{lemma2.7}.

Дальше повторяем рассуждения, приведённые в доказательсте леммы 1.9~\eqref{lemma1.9} с заменой $ x - t $ на $ t - x $, $ e^{-r(x-t)} $ на $ e^{r(x-t)} $, степеней $ x $ в подстановках - на степени $ 1 - x $ и отрезок $ [\varepsilon ,1] $ - на отрезок $ [0,1-\varepsilon] $.

Тогда придём к следующему:
\begin{equation}
\begin{array}{c}
\nonumber

D^m\Omega_{2r}^kk = r^k \int_x^1 D^m K_{20}(x,t,k,r)u(t)dt, \\
\int_x^1 D^mK_{20}(x,t,k,r)u(t)dt = \int\limits_x^1 D(D^{m-1}K_{20}(x,t,k,r))u(t)dt = \\
= - \int\limits_x^1 \dfrac{d}{dt}(D^{m-1}K_{20}(x,t,k,r))u(t)dt = - [D^{m-1}K_{20}(x,t,k,r)]_x^1 + \\ + \int\limits_x^1 D^{m-1}K_{20}(x,t,k,r)u'(t)dt = -[D^{m-1}K_{20}(x,t,k,r)]_{t=1}u(1) + \\ = \int\limits_x^1 D^{m-1}K_{20}(x,t,k,r)u'(t)dt = \\
= -[D^{m-1}K_{20}(x,t,k,r)]_{t=1}u(1) - [D^{m-2}K_{20}(x,t,k,r)]_{t=1}u'(1) - \\ - [D^2K_{20}(x,t,k,r)]_{t=1}u^{(m-3)}(1) - [DK_{20}(x,t,k,r)]_{t=1}u^{(m-2)}(1) - \\ - K_{20}(x,t,k,r)u^{(m-1)}(1) + \int\limits_x^1 K_{20}(x,t,k,r)u^{(m)}(t)dt.

\end{array}
\end{equation}

Так же как для операторов $ D^m\Omega_{1r}^k $ все подстановки на отрезке $ [0,1-\varepsilon] $ будут иметь оценку $ O(r^{m-1}e^{-r\varepsilon}) $.

Отсюда приходим к равенству:
\begin{equation}
\begin{array}{c}
\nonumber

\Vert D^m\Omega_{2r}^ku - u^{(m)} \Vert_{C[0,1-\varepsilon]} = \Vert \Omega_{2r}^ku^{(m)} - u^{(m)} \Vert_{C[0,1-\varepsilon]} + O(r^{k+m-1}e^{-r\varepsilon}),

\end{array}
\end{equation}
откуда следует утверждение леммы 2.9~\eqref{lemma2.9}.

\chapter{Восстановление функции вместе с её производными}
\section{Приближение функции и её производных на $ [0,1] $ с помощью оператора резольвенты}
С помощью оператора $ \Omega_{1r} $ из главы 1 и оператора $ \Omega_{2r} $ из главы 2 а так же их степеней можно получить приближение к непрерывной функции и её производным во внутренних точках отрезка $ [0,1] $. Теперь можно построить оператор, позволяющий получить приближение к непрерывной функции и её производным на всём отрезке.

Рассмторим оператор $ \Omega_r $, являющийся комбинацией операторов $ \Omega_{1r}, \Omega_{2r} $.

\begin{equation}
\begin{array}{c}

\Omega_r u = \left\{
\begin{array}{l}
\Omega_{2r}u \equiv r\int\limits_x^1 e^{r(x-t)}u(t)dt, x \in [0,1/2], \\\\
\Omega_{1r}u \equiv r\int\limits_0^x e^{-r(x-t)}u(t)dt, x \in [1/2,1].
\end{array}
\right.

\end{array}
\end{equation}

В силу свойств операторов $ \Omega_{1r} $ и $ \Omega_{2r} $ (леммы 1.2~\eqref{lemma1.2} и 2.2~\eqref{lemma2.2}) на каждом из отрезков $ [0,1/2] $ и $ [1/2,1] $ эти операторы дают равномерную сходимость в метрике $ C[0,1/2] $ и $ C[1/2,1] $ соответственно к любой функции $ u(x) \in C[0,1] $.

Будем смотреть на функцию $ \Omega_r u $ как на элемент пространства $ L_\infty[0,1] $, с нормой:

\begin{equation}
\begin{array}{c}

\Vert v(x)\Vert_{L_\infty[0,1]} = max\lbrace \Vert v(x) \Vert_{C[0,1/2]}, \Vert v(x) \Vert_{C[1/2,1]} \rbrace.

\end{array}
\end{equation}

Далее определим по аналогии с (3.1) в соответствии с леммами 1.5~\eqref{lemma1.5} и 2.5~\eqref{lemma2.5} оператор $ \Omega_r^{(k)} $:
\begin{equation}
\begin{array}{c}

\Omega_r^{(k)} u = \left\{
\begin{array}{l}
\Omega_{2r}^ku \equiv r^k\int\limits_x^1 \dfrac{(t-x)^{k-1}}{(k-1)!} e^{r(x-t)}u(t)dt, x \in [0,1/2], \\\\
\Omega_{1r}^ku \equiv r^k\int\limits_0^x \dfrac{(x-t)^{k-1}}{(k-1)!} e^{-r(x-t)}u(t)dt, x \in [1/2,1].
\end{array}
\right.

\end{array}
\end{equation}

А также построим оператор $ D^k\Omega_r (D^k=\dfrac{d^k}{dx^k}, D' \equiv D) $:
\begin{equation}
\begin{array}{c}

D^k\Omega_r u = \left\{
\begin{array}{l}
D^k\Omega_{2r}u, x \in [0,1/2], \\
D^k\Omega_{1r}u, x \in [1/2,1],
\end{array}
\right.
k=1,2,...

\end{array}
\end{equation}
где в соответствии с леммами 1.3~\eqref{lemma1.3} и 2.3~\eqref{lemma2.3} операторы $ D^k\Omega_{1r}u $ и $ D^k\Omega_{2r}u $ определены в формулах (1.9) и (2.8) соответственно.

Из (3.2),(3.4) и лемм 1.2~\eqref{lemma1.2},2.2~\eqref{lemma2.2},1.4~\eqref{lemma1.4} и 2.4~\eqref{lemma2.4} вытекает теорема:

\label{theorem3.1}
\textbf{Теорема 3.1}
\textit{Для любой функции $ u(x) \in C^l[0,1], l \geq 0 $ выполняется сходимость:}
\begin{equation}
\begin{array}{c}

\Vert D^k\Omega_r u - u^{(k)} \Vert_{L_\infty[0,1]} \rightarrow 0 $ \textit{при} $ r \rightarrow \infty, k =0,1,...,l.

\end{array}
\end{equation}

Рассмотрим операторы $ D^m\Omega_r^{(k)} $ при $ k \geq 1, m = 0,...,k-1 $:
\begin{equation}
\begin{array}{c}

D^m\Omega_r^{(k)} u = \left\{
\begin{array}{l}
D^m\Omega_{2r}^ku \equiv r^k\int\limits_x^1 K_{2m}(x,t,k,r) u(t)dt, x \in [0,1/2], \\\\
D^m\Omega_{1r}^ku \equiv r^k\int\limits_0^x K_{1m}(x,t,k,r) u(t)dt, x \in [1/2,1].
\end{array}
\right.


\end{array}
\end{equation}
где $ K_{1m}(x,t,k,r) $, $ K_{2m}(x,t,k,r) $ определены в формулах (1.22),(2.19) в соответствии с леммами 1.8~\eqref{lemma1.8} и 2.8~\eqref{lemma2.8}.

Из (3.2),(3.6) и лемм 1.7~\eqref{lemma1.7},2.7~\eqref{lemma2.7},1.9~\eqref{lemma1.9} и 2.9~\eqref{lemma2.9} вытекает теорема:

\label{theorem3.2}
\textbf{Теорема 3.2}
\textit{Для любой функции $ u(x) \in C^{k-1}[0,1] $ при $ k \geq 1, m = 0,...,k-1 $ выполняется сходимость:}
\begin{equation}
\begin{array}{c}

\Vert D^m\Omega_r^{(k)} u - u^{(m)} \Vert_{L_\infty[0,1]} \rightarrow 0 $ \textit{при} $ r \rightarrow \infty .

\end{array}
\end{equation}

\section{Постановка задачи восстановления функции}
Пусть $ u(x) \in C^m[0,1] $ задана приближением $ f_\delta(x) $ по метрике пространства $ L_2[0,1] $, т.е. $ \Vert f_\delta -u \Vert_{L_2[0,1]} \leq \delta $. Ставится задача по $ f_\delta $ и $ \delta $ найти равномерное приближение $ u(x) $.
Строится приближение с помощью оператора $ \Omega_r $, опредённого в (3.1).

\label{theorem3.3}
\textbf{Теорема 3.3}
\textit{Для сходимости}
\begin{equation}
\begin{array}{c}

\Delta(\delta, \Omega_r, u) \equiv \sup\limits_{f_\delta} \lbrace \Vert \Omega_r f_\delta - u \Vert_{L_\infty[0,1]}: \Vert f_\delta - u \Vert_{L_2[0,1]} \leq \delta \rbrace \rightarrow 0 $ \textit{при} $ \delta \rightarrow 0

\end{array}
\end{equation}
\textit{необходимо и достаточно выполенение согласования:}
\begin{equation}
\begin{array}{c}
\nonumber

r = r(\delta), r(\delta) \rightarrow \infty, (r(\delta))^{1/2}\delta \rightarrow 0.

\end{array}
\end{equation}

\textbf{Доказательство.} По теореме 3.1~\eqref{theorem3.1} для сходимости (3.8) необходимо и достаточно:
\begin{equation}
\begin{array}{c}
\nonumber

\Vert \Omega_r u - u \Vert_{L_\infty[0,1]} \rightarrow 0, r \rightarrow \infty $ и $ \delta \Vert \Omega_r \Vert_{L_2[0,1] \rightarrow L_\infty[0,1]} \rightarrow 0, r \rightarrow \infty , \delta \rightarrow 0.

\end{array}
\end{equation}
Первое из этих условий вытекает из теоремы 3.1~\eqref{theorem3.1}.
Для доказательства второго покажем, что $ \Vert \Omega_r \Vert_{L_2[0,1] \rightarrow L_\infty[0,1]} $ ограничена при каждом фиксированном $ r $.

В нашем случае имеем:
\begin{equation}
\begin{array}{c}

\Vert \Omega_r u - u \Vert_{L_\infty[0,1]} = \max{\lbrace\Vert \Omega_r u - u \Vert_{C[0,1/2]},\Vert \Omega_r u - u \Vert_{C[1/2,1]}\rbrace} \\
\Vert \Omega_r \Vert_{L_2[0,1] \rightarrow L_\infty[0,1]} = \max{\lbrace \Vert \Omega_r \Vert_{L_2[0,1] \rightarrow C[0,1/2]} , \Vert \Omega_r \Vert_{L_2[0,1] \rightarrow C[1/2,1]} \rbrace}

\end{array}
\end{equation}

Если $ K $ - интегральный оперетор:
\begin{equation}
\begin{array}{c}
\nonumber

Ku = \int\limits_0^1 K(x,t)u(t)dt, K:L_2[0,1] \rightarrow C[a,b], a \leq x \leq b, 0 \leq a < b \leq 1,

\end{array}
\end{equation}
то
\begin{equation}
\begin{array}{c}

\Vert K \Vert_{L_2[0,1] \rightarrow C[a,b]} = \max\limits_{x \in [a,b]}\biggl(\int\limits_0^1 K^2(x,t)dt\biggr)^{1/2}

\end{array}
\end{equation}
В соответствии с (3.1) и (3.10) имеем:
\begin{equation}
\begin{array}{c}
\nonumber

\Vert \Omega_{2r} \Vert_{L_2[0,1] \rightarrow C[0,1/2]} = r \max\limits_{0 \leq x \leq 1/2} \biggl(\int\limits_x^1 e^{2r(x-t)}dt\biggr)^{1/2}, \\\\
\int\limits_x^1 e^{2r(x-t)}dt = -\dfrac{1}{2r}e^{2r(x-t)}\biggl|_x^1 = \dfrac{1}{2r}(1-e^{-2r(1-x)}), \\
\max\limits_{0 \leq x \leq 1/2}(1-e^{-2r(1-x)}) = 1 - e^{-2r} \Rightarrow \Vert \Omega_{2r} \Vert_{L_2[0,1] \rightarrow C[0,1/2]} = \dfrac{r^{1/2}}{\sqrt{2}}( 1 - e^{-2r})^{1/2}, \\
\max\limits_{0\leq x \leq 1/2}(1-e^{-2r(1-x)}) = 1 + (1-e{-2r}) - 1 = \\\\ = 1 + \dfrac{((1-e^{-2r})^{1/2}-1)((1-e^{-2r})^{1/2}+1)}{(1-e^{-2r})^{1/2}+1} = \\\\ = 1 + \dfrac{1-e^{-2r}-1}{(1-e^{-2r})^{1/2}+1} = 1 - \dfrac{e^{-2r}}{(1-e^{-2r})^{1/2}+1} = 1 + O(e^{-2r}).

\end{array}
\end{equation}
\begin{equation}
\begin{array}{c}

\Vert \Omega_{2r} \Vert_{L_2[0,1] \rightarrow C[0,1/2]} = \dfrac{r^{1/2}}{\sqrt{2}} + O(r^{1/2}e^{-2r})

\end{array}
\end{equation}
Аналогичным образом из (3.1) получаем:
\begin{equation}
\begin{array}{c}
\nonumber

\Vert \Omega_{1r} \Vert_{L_2[0,1] \rightarrow C[1/2,1]} = r \max\limits_{1/2 \leq x \leq 1} \biggl(\int\limits_0^x e^{-2r(x-t)}dt\biggr)^{1/2}, \\\\
\int\limits_0^x e^{-2r(x-t)}dt = \dfrac{1}{2r}e^{-2r(x-t)}\biggl|_0^x = \dfrac{1}{2r}(1-e^{-2rx}), \\\\
\max\limits_{1/2 \leq x \leq 1}(1-e^{-2rx}) = 1 - e^{-2r}

\end{array}
\end{equation}

\begin{equation}
\begin{array}{c}

\Vert \Omega_{1r} \Vert_{L_2[0,1] \rightarrow C[1/2,1]} = \dfrac{r^{1/2}}{\sqrt{2}} + O(r^{1/2}e^{-2r})

\end{array}
\end{equation}

Из формул (3.9), (3.11) и (3.12) получаем:
\begin{equation}
\begin{array}{c}

\Vert \Omega_{r} \Vert_{L_2[0,1] \rightarrow L_\infty[0,1]} = \dfrac{r^{1/2}}{\sqrt{2}} + O(r^{1/2}e^{-2r}),


\end{array}
\end{equation}
а из (3.13) следует утверждение теоремы~3.3~\eqref{theorem3.3}.

\section{Постановка задачи восстановления производной функции порядка $ m $}
Пусть $ u(x) \in C^{k-1}[0,1] $ задана приближением $ f_\delta(x) $ по метрике пространства $ L_2[0,1] $.
Ставится задача по $ f_\delta $ и $ \delta $ найти равномерное приближение $ u^{(m)}(x), 0 \leq m \leq k-1 $.
Строится приближение с помощью оператора $ D^m\Omega_r^{(k)} $, опредённого в (3.6).

\label{theorem3.4}
\textbf{Теорема 3.4}
\textit{Для сходимости}
\begin{equation}
\begin{array}{c}

\Delta(\delta, D^m\Omega_r^{(k)}, u) \equiv \\ \equiv \sup\limits_{f_\delta} \lbrace \Vert D^m\Omega_r^{(k)} f_\delta - u^{(m)} \Vert_{L_\infty[0,1]}: \Vert f_\delta - u \Vert_{L_2[0,1]} \leq \delta \rbrace \rightarrow 0 $ \textit{при} $ \\
\delta \rightarrow 0, k \geq 1, 0 \leq m \leq k-1,

\end{array}
\end{equation}
\textit{необходимо и достаточно выполенение согласования:}
\begin{equation}
\begin{array}{c}
\nonumber

r = r(\delta), r(\delta) \rightarrow \infty, (r(\delta))^{\frac{2m+1}{2}}\delta \rightarrow 0.

\end{array}
\end{equation}

\textbf{Доказательство.} При $ k = 1 $ и $ m = 0 $ теорема 3.4~\eqref{theorem3.4} эквивалентна теореме 3.3~\eqref{theorem3.3}.

Так как для опереторов $ D^m \Omega_r^{(k)} $ по теореме 3.2~\eqref{theorem3.2} выполняется сходимость $ \Vert D^m \Omega_r^{(k)}u -u^{(m)} \Vert_{L_{\infty[0,1]} \rightarrow 0} $ при $ r \rightarrow \infty $, то необходимо доказать ограниченность $ \Vert D^m \Omega_r^{(k)} \Vert_{L_2[0,1] \rightarrow L_\infty[0,1]} $ при каждом фиксированном $ r $.

Для нахождения согласования $ r(\delta) $ необходимо найти асимптотику по $ r $ при $ r \rightarrow \infty $ для указанных норм:
\begin{equation}
\begin{array}{c}
\nonumber

\Vert D^m \Omega_r^{(k)} \Vert_{L_2[0,1] \rightarrow L_\infty[0,1]} = \max\lbrace \Vert D^m \Omega_{2r}^{(k)} \Vert_{L_1[0,1] \rightarrow C[0,1/2]} , \Vert D^m \Omega_{1r}^{(k)} \Vert_{L_2[0,1] \rightarrow C[1/2,1]} \rbrace

\end{array}
\end{equation}

Рассмотрим сначала норму: $ \Vert D^m \Omega_{1r}^{(k)} \Vert_{L_2[0,1] \rightarrow C[1/2,1]} $.

Из формул (1.21), (1.22) и (3.10) получаем:
\begin{equation}
\begin{array}{c}

\Vert D^m \Omega_{1r}^{(k)} \Vert_{L_2[0,1] \rightarrow C[1/2,1]} = r^k \max\limits_{1/2 \leq x \leq 1} \biggl(\int\limits_0^x K_{1m}^2(x,t,k,r)\biggr)^{1/2}

\end{array}
\end{equation}
где
\begin{equation}
\begin{array}{c}

K_{1m}^2(x,t,k,r) = e^{-2r(x-t)}\biggl[ r^m\dfrac{(x-t)^{k-1}}{(k-1)!} - nr^{m-1}\dfrac{(x-t)^{k-2}}{(k-2)!} + \\ + C_m^2r^{m-2}\dfrac{(x-t)^{k-3}}{(k-3)!} + ... + (-1)^{m-1}C_m^{m-1}r\dfrac{(x-t)^{k-m}}{(k-m)!} + \\ + (-1)^m\dfrac{(x-t)^{k-m-1}}{(k-m-1)!} \biggr]^2 = e^{-2r(x-t)}[r^2(x-t)^2-2r(x-t)+1]

\end{array}
\end{equation}
Пусть сначала $k = 2, m = 1 $. Тогда с соответствии с (3.16):
\begin{equation}
\begin{array}{c}
\nonumber

K_{11}^2(x,t,k,r)=e^{-2r(x-t)}[r(x-t)-1]^2 = e^{-2r(x-t)}[r^2(x-t)^2 - 2r(x-t) + 1]

\end{array}
\end{equation}
Сделаем замену: $ x-t=\tau $. Тогда:
\begin{equation}
\begin{array}{c}
\nonumber
K_{11}^2(\tau,2,r)=1 - 2r\tau + r^2\tau  \\
I = \int\limits_0^x K_{11}^2(x,t,2,r)dt = \int\limits_0^x K_{11}^2(\tau,2,r)d\tau \\
I = I_0 -2rI_1 + r^2I_2, $ где $ \\
I_0 = \int\limits_0^x e^{-2r\tau}d\tau , I_1 = \int\limits_0^x \tau e^{-2r\tau}d\tau , I_2 = \int\limits_0^x \tau^2 e^{-2r\tau}d\tau . \\
\end{array}
\end{equation}
Оценим эти интегралы:
\begin{equation}
\begin{array}{c}
I_0 = -\dfrac{1}{2r}(e^{-2rx}-1) = \dfrac{1}{2r}(1-e^{-2rx}), \\
I_1 = - \dfrac{1}{2r}\tau e^{-2r\tau}\int\limits_0^x d\tau + \dfrac{1}{2r}\int\limits_0^x e^{-2rx}d\tau = -\dfrac{x}{2r}e^{-2rx} + \dfrac{1}{2r}I_0, \\
I_2 = -\dfrac{1}{2r}\tau e^{-2r\tau} \biggl\vert_0^x + \dfrac{1}{2r}\int\limits_0^x 2\tau e^{-2r\tau}d\tau = -\dfrac{x^2}{2r}e^{-2rx} + \dfrac{1}{r}I_1 = \\ = -\dfrac{x^2}{2r}e^{-2rx} + \dfrac{1}{r}\biggl( -\dfrac{x}{2r}e^{-2rx} + \dfrac{1}{2r}I_0 \biggr).

\end{array}
\end{equation}
Так как $ x \in [1/2, 1] $, то:
\begin{equation}
\begin{array}{c}
\nonumber

I_0 = \dfrac{1}{2r} + O(\dfrac{1}{r}e^{-r}), \\
I_1 = O(\dfrac{1}{r}e^{-r}) + \dfrac{1}{2r}\biggl( \dfrac{1}{2r} + O(\dfrac{1}{r}e^{-r}) \biggr) = \dfrac{1}{4r^2} + O(\dfrac{1}{r}e^{-r}), \\
I_2 = O(\dfrac{1}{r}e^{-r}) + \dfrac{1}{r}\biggl( \dfrac{1}{4r^2} + O(\dfrac{1}{r}e^{-r}) \biggr) = \dfrac{1}{4r^3} + O(\dfrac{1}{r}e^{-r}).

\end{array}
\end{equation}
Получим:
\begin{equation}
\begin{array}{c}
\nonumber

I = \dfrac{1}{2r} + O(\dfrac{1}{r}e^{-r}) - 2r\biggl( \dfrac{1}{4r^2} + O(\dfrac{1}{r}e^{-r}) \biggr) + r^2\biggl( \dfrac{1}{4r^3} + O(\dfrac{1}{r}e^{-r}) \biggr) = \\ = \dfrac{1}{4r} + O(re^{-r}) = \dfrac{1}{4r}(1 + O(r^2e^{-r}))

\end{array}
\end{equation}  
\begin{equation}
\begin{array}{c}
\nonumber

I^{1/2} = \dfrac{1}{2\sqrt{r}}(1 + O(r^2e^{-r}))^{1/2}

\end{array}
\end{equation}  
Так как:
\begin{equation}
\begin{array}{c}
\nonumber

(1 + O(r^2e^{-r}))^{1/2} = 1 + O(r^2e^{-r})

\end{array}
\end{equation}  
то:
\begin{equation}
\begin{array}{c}
\nonumber

I^{1/2} = \dfrac{1}{2\sqrt{r}}(1 + O(r^2e^{-r}))

\end{array}
\end{equation}
Из (3.15) получаем:
\begin{equation}
\begin{array}{c}
\nonumber

\Vert D^m \Omega_{1r}^2 \Vert_{L_2[0,1] \rightarrow C[1/2,1]} = r^2\dfrac{1}{2\sqrt{r}}(1 + O(r^2e^{-r})) = \dfrac{1}{2} r^{3/2}(1 + O(r^2e^{-r})).

\end{array}
\end{equation}

Рассмотрим $ \Vert D^m \Omega_{12}^k \Vert_{L_2[0,1] \rightarrow C[1/2,1]} $, где $ k > 2, m = 0,...,k-1 $. Раскроем скобки, стоящие в правой части (3.хз) и выполним замены $ x-t = \tau , k - 1 = k_1 $:
\begin{equation}
\begin{array}{c}
\nonumber

K_{1m}^2(x,t,k,r) = P_{1m}(\tau ,r) $ ,где $ P_{1m}(\tau ,r) $ имеет вид:$ \\\\
P_{1m}(\tau ,r) = a_0\tau^{2k_1-2m} + a_1r\tau^{2k_1-2m+1} + a_2r^2\tau^{2k_1-2m+2} + \\ + ... + a_lr^l\tau^{2k_1-2m+l} + a_{2m}r^{2m}\tau^{2k_1}

\end{array}
\end{equation}
где $ a_j, j = 0,...,2m $ -константы, выраженные через произведение констант, стоящих при степенях $ r^{m-l}(x-t)^{k-l-1}, l = 0,...,m $ в квадратной скобке выражения (3.16).

Тогда интеграл $ I = \int\limits_0^x K_{1m}^2(x,t,k,r)dt, $, стоящий в правой части (3.15) можно представить в виде:
\begin{equation}
\begin{array}{c}

I = a_0I_0 + a_1rI_1 + ... + a_lr^lI_l + ... + a_{2m}r^{2m}I_{2m},

\end{array}
\end{equation}
где:
\begin{equation}
\begin{array}{c}
\nonumber

I_0 = \int\limits_0^x \tau^{2k_1-2m}e^{-2r\tau}d\tau , \\
I_1 = \int\limits_0^x \tau^{2k_1-2m+1}e^{-2r\tau}d\tau \\
... \\
I_l = \int\limits_0^x \tau^{2k_1-2m+l}e^{-2r\tau}d\tau \\
... \\
I_{2m} = \int\limits_0^x \tau^{2k_1}e^{-2r\tau}d\tau

\end{array}
\end{equation}
тогда мы можем записать:
\begin{equation}
\begin{array}{c}
\nonumber

I_l = \dfrac{(2k_1 -2m +l)!}{(2r)^{2k_1-2m+l+1}} + O_e

\end{array}
\end{equation}
где $ l = 0,...,2m $, а $ O_e $ - слагаемые, содержащие $ e^{-2rx} $.
Так, как $ x \in [1/2,1] $, то сумма указанных слагаемых: $ O_e = O(r^{-1}e^{-r}) $. Таким образом мы получим:
\begin{equation}
\begin{array}{c}

I_l = \dfrac{(2k_1 -2m + l)!}{(2r)^{2k_1-2m+l+1}} + O(r^{-1}e^{-r})

\end{array}
\end{equation}
где:
\begin{equation}
\begin{array}{c}
\nonumber

I_0 = \dfrac{(2k_1 -2m)!}{(2r)^{2k_1-2m+1}} + O(r^{-1}e^{-r}) \\
... \\
I_{2m} = \dfrac{(2k_1)!}{(2r)^{2k_1+1}} + O(r^{-1}e^{-r})

\end{array}
\end{equation}
Подставим в (3.18):
\begin{equation}
\begin{array}{c}
\nonumber

r^lI_l = \dfrac{(2k_1 -2m + l)!}{2^{2k_1-2m+l+1}r^{2k_1-2m+1}} + O(r^{l-1}e^{-r}), l = 0,...,2m

\end{array}
\end{equation}

Из всех слагаемых $ O(r^{l-1}e^{-r}), l = 0,...,2m $ самое большое при $ l = 2m $, т.е $ O(r^{2m-1}e^{-r}) $. Запишем интеграл $ I $ в следующем виде:
\begin{equation}
\begin{array}{c}
\nonumber

I = C\dfrac{r^{-(2k-2m+1)}}{1} + O(r^{2m-1}e^{-r}), $ где $ C = \sum\limits_{c=0}^{2m}c_l, \\
c_l = \dfrac{a_l}{2^{2k_1 - 2m + l + 1}}(2k_1 - 2m + l)!

\end{array}
\end{equation}
тогда:
\begin{equation}
\begin{array}{c}
\nonumber

I = cr^{-(2k_1 - 2m + 1)}(1 + O(r^{2k_1}e^{-r}))

\end{array}
\end{equation}
Тогда получим:
\begin{equation}
\begin{array}{c}

I^{1/2} = c^{1/2}r^{-(k_1 - m + 1/2)}(1 + O(r^{2k_1-r}))

\end{array}
\end{equation}
При этом $ c \neq 0 $. Из (3.15) и (3.20) получаем:
\begin{equation}
\begin{array}{c}
\nonumber

\Vert D^m \Omega_{1r}^k \Vert_{L_2[0,1] \rightarrow C[1/2,1]} = c^{1/2}r^{m+ 1/2}(1 + O(r^{2k-1}e^{-r}))

\end{array}
\end{equation}
Получим оценку:
\begin{equation}
\begin{array}{c}

\Vert D^m \Omega_{1r}^k \Vert_{L_2[0,1] \rightarrow C[1/2,1]} = c^{1/2}r^{\frac{2m+1}{2}} + O(r^{2k-m-3/2}e^{-r})

\end{array}
\end{equation}
Мы получили оценку для нормы $ \Vert D^m \Omega_{1r}^k \Vert_{L_2[0,1] \rightarrow C[1/2,1]} $.

Теперь рассмотрим норму: $ \Vert D^m \Omega_{2r}^{(k)} \Vert_{L_1[0,1] \rightarrow C[0,1/2]} $.

Для получения оценки проведём те же рассуждения что в первой части доказательства с заменой (1.21) из леммы 1.8~\eqref{lemma1.8} на (2.18) из леммы 2.8~\eqref{lemma2.8}, $ K_{1m}(x,t,k,r) $ на $ K_{2m}(x,t,k,r) $, приведённое в (2.19) из леммы 2.8~\eqref{lemma2.8}, $ x - t = t - x $, $ \int\limits_0^x $ на $ \int\limits_x^1 $, $ \tau = t - x $ на $ \tau = x - t $, а $ e^{-2rx} $ на $ e^{-2r(1-x)} $.

Учитывая, что $ 0 \leq x \leq \dfrac{1}{2} $ при вычислении $ I_l $ имеет вид:
\begin{equation}
\begin{array}{c}
\nonumber

I_l = \int\limits_x^1 e^{-2r\tau}\tau^{2k_1 - 2m + 1}d\tau

\end{array}
\end{equation}
мы придём к выражению, стоящему в правой части выражения (3.19). Для сходимости оператора справедливо (3.21) с заменой $ \Omega_{1r} $ на $ \Omega_{2r} $. 

Отсюда справедлива оценка:
\begin{equation}
\begin{array}{c}

\Vert D^m \Omega_r^{(k)} \Vert_{L_2[0,1] \rightarrow L_\infty[0,1]} = c^{1/2}r^{\dfrac{2m+1}{2}} + O(r^{2k-m-3/2}e^{-r})

\end{array}
\end{equation}
Из выражений (3.9), (3.22) следует что константа $ c \neq 0 $, в противном случае норма оператора $ \rightarrow 0 $ при $ r \rightarrow \infty $ вне зависимости от $ r $.

$ D^m\Omega_r^{k} $ апроксимирует неограниченный оператор. Обозначим для краткости $ B_r = D^m\Omega_r^{k} $.

Покажем, что последовательность $B_ru$ сходится в $ L_\infty $ для любой функции $ u \in L_2 $, т.е. $ u(x) \in C^m[0,1] $.

Рассмотрим $ u_\varepsilon \in C^m[0,1] $, такой чтобы $ \Vert u_\varepsilon - u \Vert_{L_2[0,1]} \leq \varepsilon $
\begin{equation}
\begin{array}{c}
\nonumber

\Vert B_{r'}u - B_{r''}u \Vert_{L_\infty} \leq \Vert (B_{r'} - B_{r''})u_\varepsilon \Vert_{L_\infty} + \\ + \Vert (B_{r'} - B_{r''})(u-u_\varepsilon) \Vert_{L_\infty} \leq \Vert (B_{r'} - B_{r''})u_\varepsilon \Vert_{L_\infty} + 2K\varepsilon

\end{array}
\end{equation}
где $ K $ ограниченный оператор при $ C \neq 0 $. Последовательность $ B_ru_\varepsilon $ сходится по теореме 3.2~\eqref{theorem3.2}, значит она является фундаментальной, а отсюда и из приведённой выше оценки получаем, что $ B_ru $ сходится при $ r \rightarrow \infty $ и по теореме Банаха-Штейнгауса $ B $ ограничен.
\begin{equation}
\begin{array}{c}
\nonumber

Bu = D^mu, u \in C^m[0,1] \\
\Vert D^mu \Vert_{L_\infty} < K\Vert u \Vert_{L_2} \forall u \in C^m

\end{array}
\end{equation}
чего не может быть.

Равенство (3.22) и теорема 3.2~\eqref{theorem3.2} приводят к утверждению теоремы 3.4~\eqref{theorem3.4}

\conclusions

В данной работе, отправляясь от резольвенты простейшего дифференциального оператора первого порядка, построен оператор, позволяющий равномерно аппроксимировать непрерывные функции и их производные любого порядка на отрезках $ [0,\varepsilon] $ и $ [\varepsilon,1] $. Далее задача была расширена на отрезок $ [0,1] $, в результате чего были построены семейства интегральных операторов, которые можно использовать для аппроксимации решений некорректно поставленных задач. Далее были поставлены задачи восстановления функции и задачи восстановления производной функции порядка $ m $, далее эти задачи были решены с использованием построенных семейств интегральных операторов. 

Решённые задачи позволяют находить решения для неустойчивых (некорректных) задач. Данные выводы очень важны, так как неустойчивые (некорректные) задачи возникают при описании многих физических явлений: в геофизике, спектроскопии,  астрофизике и т.д., а также в теоретических исследованиях, например, в теории приближений. В связи с развитием необходимости решения данных задач нельзя недооценить целесообразность развития методов их решения.

%\biblio
%
%\appendix
%\makeatletter
%  \gdef\thechapter{\@Asbuk\c@chapter}
%\makeatother
%\singlespacing


\end{document}
