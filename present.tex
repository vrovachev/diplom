\documentclass[a4paper,14pt,russian]{extreport}

\hoffset 0pt
\voffset 0pt

\usepackage[
  a4paper, includehead, includefoot, mag=1000,
  headsep=0mm, headheight=0mm,
  left=25mm, right=15mm, top=20mm, bottom=20mm
]{geometry}

\usepackage[T2A]{fontenc}
\usepackage[utf8]{inputenc}
\usepackage[russian]{babel}

% \usepackage{cmap} % Улучшенный поиск русских слов в полученном pdf-файле
\usepackage[unicode, pdftex]{hyperref}
\usepackage{pdfpages}
\usepackage[nottoc]{tocbibind}

\usepackage[onehalfspacing]{setspace} %"умное" расстояние между строк - установить 1.5 интервала от нормального
\usepackage{cite}  %"умные" библиографические ссылки (сортировка и сжатие)
\usepackage{indentfirst} %делать отступ в начале параграфа
\usepackage{enumerate}  %создание и автоматическая нумерация списков
\usepackage{longtable} % Длинные таблицы
\usepackage{multirow,makecell,array} % Улучшенное форматирование таблиц
\usepackage{graphicx} \graphicspath{{images/}}
\usepackage{pdflscape} % Для включения альбомных страниц
\renewcommand{\rmdefault}{ftm} % Включаем Times New Roman
%%% Выравнивание и переносы %%%
\sloppy % Избавляемся от переполнений
\clubpenalty=10000 % Запрещаем разрыв страницы после первой строки абзаца
\widowpenalty=10000 % Запрещаем разрыв страницы после последней строки абзаца
\righthyphenmin=2 % Минимальное число символов при переносе - 2.

\usepackage{fancyvrb}

\usepackage{amssymb,amsmath,amsfonts,latexsym,mathtext} %расширенные наборы  математических символов

\usepackage{amsthm}
\theoremstyle{definition}
\newtheorem{theorem}{Теорема}
\newtheorem{proposition}[theorem]{Предложение}
\newtheorem{corollary}[theorem]{Следствие}
\newtheorem{lemma}[theorem]{Лемма}
\newtheorem{definition}[theorem]{Определение}
\newtheorem{example}[theorem]{Пример}
\newtheorem{remark}[theorem]{Замечание}

\usepackage[tableposition=top]{caption}
\DeclareCaptionLabelFormat{gostfigure}{Рисунок #2}
\DeclareCaptionLabelFormat{gosttable}{Таблица #2}
\DeclareCaptionLabelSeparator{gost}{~---~}
\captionsetup{labelsep=gost}
\captionsetup[figure]{labelformat=gostfigure}
\captionsetup[table]{labelformat=gosttable}

\usepackage{fancyhdr}

\pagestyle{fancyplain}
\renewcommand{\headrulewidth}{0pt}
\fancyhf{}
\rfoot{\fancyplain{}{\thepage}}

\makeatletter 
\renewcommand\chapter{\if@openright\cleardoublepage\else\clearpage\fi \thispagestyle{fancyplain}%
\global\@topnum\z@ \@afterindentfalse \secdef\@chapter\@schapter} 
\makeatother

\addtocontents{toc}{\protect\thispagestyle{fancyplain}}

\setcounter{page}{1}

\usepackage{titlesec}
\titleformat{\chapter}
	{\normalsize\bfseries}
	{\thechapter}
	{1em}{}
	
\titleformat{\section}
	{\normalsize\bfseries}
	{\thesection}
	{1em}{}
	
\titleformat{\subsection}
	{\normalsize\bfseries}
	{\thesubsection}
	{1em}{}

\titleformat{\paragraph}
	{\normalsize\bfseries}
	{\thesection}
	{1em}{}

	
\titlespacing*{\chapter}{0pt}{-30pt}{8pt}
\titlespacing*{\section}{\parindent}{*4}{*4}
\titlespacing*{\subsection}{\parindent}{*4}{*4}
\titlespacing*{\paragraph}{\parindent}{*4}{*4}

\addto\captionsrussian{%
  \renewcommand\contentsname{CОДЕРЖАНИЕ}
  \renewcommand\appendixname{ПРИЛОЖЕНИЕ}
  \renewcommand\bibname{СПИСОК ИСПОЛЬЗОВАННЫХ ИСТОЧНИКОВ}
}

\usepackage{enumitem}
\makeatletter
    \AddEnumerateCounter{\asbuk}{\@asbuk}{м)}
\makeatother
\setlist{nolistsep}
\renewcommand{\labelitemi}{-}
\renewcommand{\labelenumi}{\asbuk{enumi})}
\renewcommand{\labelenumii}{\arabic{enumii})}

\usepackage{tocloft}
\renewcommand{\cfttoctitlefont}{\hspace{0.38\textwidth} \bfseries\MakeUppercase}
\renewcommand{\cftbeforetoctitleskip}{-1em}
\renewcommand{\cftaftertoctitle}{\mbox{}\hfill \\ \mbox{}\hfill{\footnotesize Стр.}\vspace{-2.5em}}
\renewcommand{\cftchapfont}{\normalsize\bfseries}
\renewcommand{\cftsecfont}{\hspace{31pt}}
\renewcommand{\cftsubsecfont}{\hspace{11pt}}
\renewcommand{\cftbeforechapskip}{1em}
\renewcommand{\cftparskip}{-1mm}
\renewcommand{\cftdotsep}{1}
\renewcommand{\cftchapdotsep}{\cftdotsep}
\setcounter{tocdepth}{2} % задать глубину оглавления — до subsection включительно

\newcommand{\likechapterheading}[1]{
    \newpage
    \begin{center}
    \textbf{\MakeUppercase{#1}}
    \end{center}}

\newcommand{\abbreviations}{\likechapterheading{ОБОЗНАЧЕНИЯ И СОКРАЩЕНИЯ}\addcontentsline{toc}{chapter}{ОБОЗНАЧЕНИЯ И СОКРАЩЕНИЯ}}
\newcommand{\definitions}{\likechapterheading{ОПРЕДЕЛЕНИЯ}\addcontentsline{toc}{chapter}{ОПРЕДЕЛЕНИЯ}}
\newcommand{\abbrevdef}{\likechapterheading{ОПРЕДЕЛЕНИЯ, ОБОЗНАЧЕНИЯ И СОКРАЩЕНИЯ}\addcontentsline{toc}{chapter}{ОПРЕДЕЛЕНИЯ, ОБОЗНАЧЕНИЯ И СОКРАЩЕНИЯ}}
\newcommand{\intro}{\likechapterheading{ВВЕДЕНИЕ}\addcontentsline{toc}{chapter}{ВВЕДЕНИЕ}}
\newcommand{\conclusions}{\likechapterheading{ЗАКЛЮЧЕНИЕ}\addcontentsline{toc}{chapter}{ЗАКЛЮЧЕНИЕ}}

\makeatletter
  \renewcommand{\@biblabel}[1]{#1.}	% Заменяем библиографию с квадратных скобок на точку:
\makeatother

\newcommand{\biblio}{
  \bibliographystyle{ugost2008}	% Оформляем библиографию в соответствии с ГОСТ 7.0.5
  \nocite{*}
  \bibliography{biblio}
}

\newcommand{\appendxchapter}[1]{ 
    \clearpage
    \stepcounter{chapter}    
    \chapter*{\appendixname~\Asbuk{chapter}\;#1}
    \addcontentsline{toc}{chapter}{\appendixname~\Asbuk{chapter}\;#1}
}
 

\renewcommand{\rmdefault}{cmr} % Шрифт с засечками
\renewcommand{\sfdefault}{cmss} % Шрифт без засечек
\renewcommand{\ttdefault}{cmtt} % Моноширинный шрифт

\begin{document}

\section{Лемма 1.1}
\label{lemma1.1}

\textit{Для $ y(x) = R_\lambda(L_1)u$ имеет место формула:}

\begin{equation}
\begin{array}{c}

y(x) \equiv R_\lambda(L_1)u = \int\limits_0^x e^{\lambda(x-t)}u(t)dt.

\end{array}
\end{equation}

\section{Лемма 1.2}
\label{lemma1.2}

\textit{Для любой функции $ u(x) \in C[0,1] $ имеет место сходимость:}

\begin{equation}
\begin{array}{c}

\Vert rR_{-r}(L_1)u-u \Vert _{C[\varepsilon ,1]} \rightarrow 0 $ \textit{при} $ r \rightarrow \infty,

\end{array}
\end{equation}

\textit{ $ \varepsilon $ - произвольное малое положительное число.}

\section{Лемма 1.3}
\label{lemma1.3}

\textit{Операторы $ D^kR_{-r}(L_1) $ имеют вид:}

\begin{equation}
\begin{array}{c}

D^kR_{-r}(L_1)u = u^{(k-1)}(x) - ru^{(k-2)}(x) + r^2u^{(k-3)}(x) - ... + \\
+ (-1)^{k-1}r^{k-1}u(x) + (-1)^kr^k\int\limits_0^x e^{-r(x-t)}u(t)dt, k = 1,...,l.

\end{array}
\end{equation}

\section{Лемма 1.4}
\label{lemma1.4}

\textit{Если $ u(x) \in C^l[0,1] $, то имеет место сходимость:}

\begin{equation}
\begin{array}{c}

\Vert rD^k R_{-r}(L_1)u -u^{(k)}(x) \Vert_{C[\varepsilon ,1} \rightarrow 0 $ \textit{при} $ r \rightarrow \infty, k = 1,...,l.

\end{array}
\end{equation}

\textit{ $ \varepsilon $ - произвольное малое положительное число.}

\section{Лемма 1.5}
\label{lemma1.5}

\textit{Операторы $ (rR_{-r}(L_1))^k $ имеют вид:}

\begin{equation}
\begin{array}{c}

(rR_{-r}(L_1))^ku = r^k\int\limits_0^x \dfrac{(x-t)^{k-1}}{(k-1)!}e^{-r(x-t)}u(t)dt.

\end{array}
\end{equation}

\section{Лемма 1.6}
\label{lemma1.6}

\textit{Если $ u \in C^1[0,1] $, то операторы $ \Omega_{1r}^k $ имеют представление:}

\begin{equation}
\begin{array}{c}

\Omega_{1r}^ku = -\dfrac{r^{k-1}x^{k-1}e^{-rx}}{(k-1)!}u(0) + \Omega_{1r}^{k-1}u - \dfrac{1}{r}\Omega_{1r}^ku',

\end{array}
\end{equation}

\textit{где $ k \geq 2 $.}

\section{Лемма 1.7}
\label{lemma1.7}

\textit{Для $ u(x) \in C[0,1] $ справедливы соотношения:}

\begin{equation}
\begin{array}{c}

\Vert \Omega_{1r}^ku - u \Vert_{C[\varepsilon ,1]} \rightarrow 0 $ \textit{при} $ r \rightarrow \infty , k = 1,2,...

\end{array}
\end{equation}

\section{Лемма 1.8}
\label{lemma1.8}

\textit{Операторы $ D^m\Omega_{1r}^ku $ при $ k \geq 1, m = 0,...,k-1 $ имеют вид}

\begin{equation}
\begin{array}{c}

D^m\Omega_{1r}^ku = r^k\int\limits_0^x K_{1m}(x,t,k,r)u(t)dt,

\end{array}
\end{equation}

\textit{где}

\begin{equation}
\begin{array}{c}

K_{1m}(x,t,k,r) = (-1)^me^{-r(x-t)} \biggl[r^m\dfrac{(x-t)^{k-1}}{(k-1)!} - mr^{m-1}\dfrac{(x-t)^{k-2}}{(k-2)!} + \\ + C_m^2r^{m-2}\dfrac{(x-t)^{k-3}}{(k-3)!} + ... + (-1)^{m-1}C_m^{m-1}r\dfrac{(x-t)^{k-m}}{(k-m)!} + \\ 
+ (-1)^m\dfrac{(x-t)^{k-m-1}}{(k-m-1)!}\biggr].

\end{array}
\end{equation}

\section{Лемма 1.9}
\label{lemma1.9}
\textit{При $ k \geq 2, m=1,...,k-1 $ для любой функции $ u(x) \in C^{k-1}[0,1] $ справедливы соотношения:}

\begin{equation}
\begin{array}{c}

\Vert D^m\Omega_{1r}^ku - u^{(m)} \Vert_{C[\varepsilon ,1]} \rightarrow 0 $ \textit{при} $ r \rightarrow \infty,

\end{array}
\end{equation}
\textit{Где $ D^m\Omega_{1r}^ku $ определены в Лемме 1.8.}

\section{Лемма 2.1}
\label{lemma2.1} 
\textit{Для $ y(x) = R_\lambda(L_2) $ имеет место формула:}
\begin{equation}
\begin{array}{c}

y(x) \equiv R_\lambda(L_2)u = -\int\limits_x^1 e^{\lambda (x-t)}u(t)dt.

\end{array}
\end{equation}

\section{Лемма 2.2}
\label{lemma2.2}
\textit{Для любой непрерывной функции $ u(x) $ имеет место сходимость:}
\begin{equation}
\begin{array}{c}

\Vert -rR_r(L_2)u - u \Vert_{C[0,1-\varepsilon]} \rightarrow 0 $ \textit{при} $ r \rightarrow \infty

\end{array}
\end{equation}
\textit{где $ \varepsilon $ - любое малое положительное число.}

\section{Лемма 2.3}
\label{lemma2.3}
\textit{Операторы $ D^kR_r(L_2) $ имеют вид:}
\begin{equation}
\begin{array}{c}

D^kR_r(L_2)u = u^{(k-1)}(x) - ru^{(k-2)}(x) + r^2u^{(k-3)}(x) + ... + \\
+ (-1)^{k-1}r^{k-1}u(x) + (-1)^kr^k\int\limits_x^1 e^{r(x-t)}u(t)dt.

\end{array}
\end{equation}

\section{Лемма 2.4}
\label{lemma2.4}
\textit{Если $ u(x) \in C^l[0,1] $, то имеет место сходимость:}
\begin{equation}
\begin{array}{c}

\Vert -rD^kR(L_2)u - u^{(k)}(x) \Vert_{C[0,1-\varepsilon]} \rightarrow 0 $ \textit{при} $ r \rightarrow \infty, k = 1,...,l.

\end{array}
\end{equation}

\section{Лемма 2.5}
\label{lemma2.5}
\textit{операторы $ \Omega_{2r}^k $ имеют вид:}
\begin{equation}
\begin{array}{c}

\Omega_{2r}^ku = r^k\int\limits_x^1 \dfrac{(t-x)^{k-1}}{(k-1)!}e^{r(x-t)}u(t)dt.

\end{array}
\end{equation}

\section{Лемма 2.6}
\label{lemma2.6}
\textit{Если $ u(x) \in C^1[0,1] $, то операторы $ \Omega_{2r}^k $ имеют представление:}
\begin{equation}
\begin{array}{c}

\Omega_{2r}^ku = -\dfrac{r^{k-1}(1-x)^{k-1}e^{-r(1-x)}}{(k-1)!}u(1) + \Omega_{2r}^{k-1}u + \dfrac{1}{r}\Omega_{2r}^ku'.

\end{array}
\end{equation}

\section{Лемма 2.7}
\label{lemma2.7}

\textit{Для $ u(x) \ in C[0,1] $ справедливы соотношения:}
\begin{equation}
\begin{array}{c}

\Vert \Omega_{2r}^ku - u \Vert_{C[0,1\varepsilon]} \rightarrow 0 $ \textit{при} $ r \rightarrow \infty, k = 1,2,...

\end{array}
\end{equation}

\section{Лемма 2.8}
\label{lemma2.8}

\textit{Операторы  имеют вид}
\begin{equation}
\begin{array}{c}

D^m\Omega_{2r}^ku = r^k\int\limits_x^1 K_{2m}(x,t,k,r)u(t)dt,

\end{array}
\end{equation}
\textit{где}
\begin{equation}
\begin{array}{c}

K_{2m}(x,t,k,r) = e^{r(x-t)} \biggl[ r^m\dfrac{(t-x)^{k-1}}{(k-1)!} - mr^{m-1}\dfrac{(t-x)^{k-2}}{(k-2)!} + C_m^2r^{m-2}\dfrac{(t-x)^{k-3}}{(k-3)!} + ... + (-1)^{m-1}C_m^{m-1}r\dfrac{(t-x)^{k-m}}{(k-m)!} + (-1)^m\dfrac{(t-x)^{k-m-1}}{(k-m-1)!}\biggr].

\end{array}
\end{equation}

\section{Лемма 2.9}
\label{lemma2.9}

\textit{При $ k \geq 2, m = 1,...,k-1 $ для любой функции $ u(x) \in C^{k-1}[0,1] $ справедливы соотношения:}
\begin{equation}
\begin{array}{c}

\Vert D^m\Omega_{2r}^ku - u^{(m)} \Vert_{C[0,1-\varepsilon]} \rightarrow 0 $ \textit{при} $ r \rightarrow \infty .

\end{array}
\end{equation}

\end{document}