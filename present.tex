% \documentclass[9pt,utf8,handout,notheorems]{beamer}
\documentclass[10pt,utf8,presentation,notheorems]{beamer}

% \usepackage{lmodern}
% \usepackage{mathptmx}
% \usepackage{helvet}

% \usepackage{pscyr}

\usepackage{ucs}
\usepackage[T2A]{fontenc}
\usepackage[utf8]{inputenc}
\usepackage[russian]{babel}

\usepackage{xtab}
\usepackage{moreverb}
\usepackage{listings}

\def\refer{\color{-red!50}}
\def\remark{\color{-green!10}}
\def\defin{\color{magenta}}
\def\comment{\color{violet}}

\mode<presentation> {
%   \usetheme{default}
  \usetheme{Boadilla}

  \usecolortheme{default}
%   \usecolortheme{crane}
%   \usecolortheme{wolverine}

%   \usefonttheme{serif}
%   \usefonttheme{structurebold}
%   \usefonttheme{structureitalicserif}
  \setbeamercovered{transparent}
  \setbeamercolor{title}{fg=red!90!black,bg=red!10!white}
}

% % for print
% \mode<handout> {
%   \hoffset 30pt
%   \voffset 0pt
%   \usecolortheme{dove}
%   \usepackage{pgfpages}
%   \pgfpagesuselayout{2 on 1}[a4paper,border shrink=5mm]
%   \pagestyle{plain}
% }

\usepackage{graphicx}
\graphicspath{{fig/}}

\usepackage{xtab}

\usepackage{amsmath}
\usepackage{amsfonts}
\usepackage{amssymb}
\usepackage{amsthm}
\usepackage{mathrsfs}


%        Курсивные окружения
\theoremstyle{plain}
\newtheorem{theorem}{\hspace*{6mm}Теорема}[section]
\newtheorem{lemma}{\hspace*{6mm}Лемма}[section]
\newtheorem{corollary}{\hspace*{6mm}Следствие}[section]
% Теорема 1.1, лемма 1.1 и т.п.

%        Ненумерованные окружения
\newtheorem*{theoremnn}{Теорема}
% Просто теорема (без номера)

%        Окружения с прямым шрифтом
\theoremstyle{definition}
\newtheorem{definition}{\hspace*{6mm}Определение}[section]

\renewenvironment{proof}
{\textbf{Доказательство. }}{\hfill$\Box$}

\renewcommand{\Re}{\mathrm{Re}\,}
\renewcommand{\Im}{\mathrm{Im}\,}

\title[Решение задачи восстановления функции]{
Решение задачи восстановления функции \\ вместе с её производными с помощью операторов $ \Omega_r $
}

\author[Ровачев В.А.]{
  Ровачев Вадим Алексеевич
}
\institute[010501] {
Специальность 010501 \\
«Прикладная математика и информатика»\\

\hfill

Научный руководитель: Хромов А.А.
}

\date{Саратов 2014}

\begin{document}

\begin{frame}
\titlepage

\end{frame}

\section<presentation>*{Содержание}

\begin{frame}
\frametitle{Содержание}
\begin{itemize}
\item 1. Постановка задачи
\item 2. Приближающие свойства резольвенты оператора \\ $ L_1:y', y(0)=0 $ на отрезке $ [\varepsilon, 1] $.
\item 3. Приближающие свойства резольвенты оператора \\ $ L_2:y', y(1)=0 $ на отрезке $ [0, 1 - \varepsilon] $.
\item 4. Восстановление функции вместе с её производными
\item 5. Заключение
\item 6. Список литературы
\end{itemize}
\end{frame}

\section{Постановка задачи}

\begin{frame}
\frametitle{Постановка задачи}
\begin{itemize}
\item Постановка задачи восстановления функции.

Пусть $ u(x) \in C^m[0,1] $ задана приближением $ f_\delta(x) $ по метрике пространства $ L_2[0,1] $, т.е. $ \Vert f_\delta -u \Vert_{L_2[0,1]} \leq \delta $. Ставится задача по $ f_\delta $ и $ \delta $ найти равномерное приближение $ u(x) $.
Строится приближение с помощью оператора $ \Omega_r $

\item Постановка задачи восстановления производной функции.

Пусть $ u(x) \in C^{k-1}[0,1] $ задана приближением $ f_\delta(x) $ по метрике пространства $ L_2[0,1] $.
Ставится задача по $ f_\delta $ и $ \delta $ найти равномерное приближение $ u^{(m)}(x), 0 \leq m \leq k-1 $.
Строится приближение с помощью оператора $ D^m\Omega_r^{(k)} $
\end{itemize}
\end{frame}

\section{Приближающие свойства резольвенты оператора $ L_1:y', y(0)=0 $ на отрезке $ [\varepsilon, 1] $.}

\begin{frame}
\frametitle{Формула резольвенты дифференциального оператора первого порядка}
Рассмотрим простейший дифференциальный оператор первого порядка $ L_1:y^{'}, y(0)=0 $. Обозначим через $ R_\lambda(L_1) $ его резольвенту, т.е. оператор $ R_\lambda(L_1)=(L_1-\lambda E)^{-1} $, где $ E $ - единичный оператор, $ \lambda $ - спектральный параметр (числовой параметр,  вообще говоря, комплексный). Найдем формулу для резольвенты.

\label{lemma1.1}
\textbf{Лемма 1.1. Формула резольвенты дифференциального оператора первого порядка}

\textit{Для $ y(x) = R_\lambda(L_1)u$ имеет место формула:}
\begin{equation}
\begin{array}{c}
y(x) \equiv R_\lambda(L_1)u = \int\limits_0^x e^{\lambda(x-t)}u(t)dt.
\end{array}
\end{equation}
Положим в (7) $ \lambda = -r $, где $ r > 0 $, и рассмотрим операторы $ rR_{-r}(L_1)$. Эти операторы имеют вид:
\begin{equation}
\begin{array}{c}
\Omega_{1r}u = rR_{-r}(L_1)u = r \int\limits_0^x e^{-r(x-t)}u(t)dt
\end{array}
\end{equation}
\end{frame}

\begin{frame}
\frametitle{Формула резольвенты в пространствах гладких функций $ C^l[0,1] $ и формула степеней резольвенты}
Рассмотрим операторы $ D^kR_{-r}(L_1) \equiv (R_{-r}(L_1)u)_x^{(k)} k = 1,...,l, D^1 \equiv D (Du = u')$.

\label{lemma1.3}
\textbf{Лемма 1.3. Формула резольвенты в пространствах гладких функций}

\textit{Операторы $ D^kR_{-r}(L_1) $ имеют вид:}
\begin{equation}
\begin{array}{c}
D^kR_{-r}(L_1)u = u^{(k-1)}(x) - ru^{(k-2)}(x) + r^2u^{(k-3)}(x) - ... + \\
+ (-1)^{k-1}r^{k-1}u(x) + (-1)^kr^k\int\limits_0^x e^{-r(x-t)}u(t)dt, k = 1,...,l.
\end{array}
\end{equation}

\label{lemma1.5}
\textbf{Лемма 1.5. Формула степеней резольвенты}

\textit{Операторы $ (rR_{-r}(L_1))^k $ имеют вид:}
\begin{equation}
\begin{array}{c}
(rR_{-r}(L_1))^ku = r^k\int\limits_0^x \dfrac{(x-t)^{k-1}}{(k-1)!}e^{-r(x-t)}u(t)dt.
\end{array}
\end{equation}
\end{frame}

\begin{frame}
\frametitle{Формула производной от резольвенты порядка $ m $}

\label{lemma1.8}
\textbf{Лемма 1.8. Формула производной от резольвенты порядка $ m $}

\textit{Операторы $ D^m\Omega_{1r}^ku $ при $ k \geq 1, m = 0,...,k-1 $ имеют вид}
\begin{equation}
\begin{array}{c}
D^m\Omega_{1r}^ku = r^k\int\limits_0^x K_{1m}(x,t,k,r)u(t)dt,
\end{array}
\end{equation}
\textit{где}
\begin{equation}
\begin{array}{c}
K_{1m}(x,t,k,r) = (-1)^me^{-r(x-t)} \biggl[r^m\dfrac{(x-t)^{k-1}}{(k-1)!} - mr^{m-1}\dfrac{(x-t)^{k-2}}{(k-2)!} + \\ + C_m^2r^{m-2}\dfrac{(x-t)^{k-3}}{(k-3)!} + ... + (-1)^{m-1}C_m^{m-1}r\dfrac{(x-t)^{k-m}}{(k-m)!} + \\ 
+ (-1)^m\dfrac{(x-t)^{k-m-1}}{(k-m-1)!}\biggr].
\end{array}
\end{equation}
\end{frame}

\section{Приближающие свойства резольвенты оператора $ L_2:y', y(1)=0 $ на отрезке $ [0, 1 - \varepsilon] $.}

\begin{frame}
\frametitle{Формула резольвенты дифференциального оператора первого порядка}
Рассмотрим оператор $ L_2: y', y(1) = 0 $, отличающийся от оператора $ L_1 $ лишь граничным условием.
Обозначим его резольвенту $R_\lambda(L_2)$. Положим $ \lambda = r, r > 0 $ и рассмотрим оператор $ -rR_r(L_2) $.

\label{lemma2.1}
\textbf{Лемма 2.1. Формула резольвенты дифференциального оператора первого порядка}

\textit{Для $ y(x) = R_\lambda(L_2) $ имеет место формула:}
\begin{equation}
\begin{array}{c}
y(x) \equiv R_\lambda(L_2)u = -\int\limits_x^1 e^{\lambda (x-t)}u(t)dt.
\end{array}
\end{equation}
Положим в (1.1) $ \lambda \ -r $, где $ r > 0 $, и рассмотрим операторы $ -rR_{r}(L_2)$. Очевидно, эти операторы имеют вид:
\begin{equation}
\begin{array}{c}
\Omega_{2r}u = -rR_{r}(L_2)u = r \int\limits_x^1 e^{r(x-t)}u(t)dt
\end{array}
\end{equation}
\end{frame}

\begin{frame}
\frametitle{Формула резольвенты в пространствах гладких функций $ C^l[0,1] $ и формула степеней резольвенты}
Рассмотрим операторы $ D^kR_r(L_2)u \equiv (R_r(L_2)u)_x^{(k)}, k = 1,...,l, D^1 \equiv D (Du = u') $.

\label{lemma2.3}
\textbf{Лемма 2.3. Формула резольвенты в пространствах гладких функций}

\textit{Операторы $ D^kR_r(L_2) $ имеют вид:}
\begin{equation}
\begin{array}{c}
D^kR_r(L_2)u = u^{(k-1)}(x) - ru^{(k-2)}(x) + r^2u^{(k-3)}(x) + ... + \\
+ (-1)^{k-1}r^{k-1}u(x) + (-1)^kr^k\int\limits_x^1 e^{r(x-t)}u(t)dt.
\end{array}
\end{equation}

\label{lemma2.5}
\textbf{Лемма 2.5. Формула степеней резольвенты}

\textit{операторы $ \Omega_{2r}^k $ имеют вид:}
\begin{equation}
\begin{array}{c}
\Omega_{2r}^ku = r^k\int\limits_x^1 \dfrac{(t-x)^{k-1}}{(k-1)!}e^{r(x-t)}u(t)dt.
\end{array}
\end{equation}
\end{frame}

\begin{frame}
\frametitle{Формула производной от резольвенты порядка $ m $}

\label{lemma2.8}
\textbf{Лемма 2.8. Формула производной от резольвенты порядка}

\textit{Операторы  имеют вид}
\begin{equation}
\begin{array}{c}
D^m\Omega_{2r}^ku = r^k\int\limits_x^1 K_{2m}(x,t,k,r)u(t)dt,
\end{array}
\end{equation}
\textit{где}
\begin{equation}
\begin{array}{c}
K_{2m}(x,t,k,r) = e^{r(x-t)} \biggl[ r^m\dfrac{(t-x)^{k-1}}{(k-1)!} - mr^{m-1}\dfrac{(t-x)^{k-2}}{(k-2)!} + \\ + C_m^2r^{m-2}\dfrac{(t-x)^{k-3}}{(k-3)!} + ... + (-1)^{m-1}C_m^{m-1}r\dfrac{(t-x)^{k-m}}{(k-m)!} + \\ + (-1)^m\dfrac{(t-x)^{k-m-1}}{(k-m-1)!}\biggr].
\end{array}
\end{equation}
\end{frame}

\section{Восстановление функции вместе с её производными}
\begin{frame}
\frametitle{Приближение функции и её производных на $ [0,1] $ с помощью оператора $ \Omega_r $}
Рассмотрим оператор $ \Omega_r $, являющийся комбинацией операторов $ \Omega_{1r}, \Omega_{2r} $ приведённых в лемме 1.1~\eqref{lemma1.1} и лемме 2.1~\eqref{lemma2.1} соответственно.
\begin{equation}
\begin{array}{c}
\Omega_r u = \left\{
\begin{array}{l}
\Omega_{2r}u \equiv r\int\limits_x^1 e^{r(x-t)}u(t)dt, x \in [0,1/2], \\
\Omega_{1r}u \equiv r\int\limits_0^x e^{-r(x-t)}u(t)dt, x \in [1/2,1].
\end{array}
\right.
\end{array}
\end{equation}
В силу свойств операторов $ \Omega_{1r} $ и $ \Omega_{2r} $ на каждом из отрезков $ [0,1/2] $ и $ [1/2,1] $ эти операторы дают равномерную сходимость в метрике $ C[0,1/2] $ и $ C[1/2,1] $ соответственно к любой функции $ u(x) \in C[0,1] $.
\end{frame}

\begin{frame}
Далее определим по аналогии с (13) в соответствии с леммами 1.5~\eqref{lemma1.5} и 2.5~\eqref{lemma2.5} оператор $ \Omega_r^{(k)} $:
\begin{equation}
\begin{array}{c}
\Omega_r^{(k)} u = \left\{
\begin{array}{l}
\Omega_{2r}^ku \equiv r^k\int\limits_x^1 \dfrac{(t-x)^{k-1}}{(k-1)!} e^{r(x-t)}u(t)dt, x \in [0,1/2], \\
\Omega_{1r}^ku \equiv r^k\int\limits_0^x \dfrac{(x-t)^{k-1}}{(k-1)!} e^{-r(x-t)}u(t)dt, x \in [1/2,1].
\end{array}
\right.
\end{array}
\end{equation}

А также построим оператор $ D^k\Omega_r (D^k=\dfrac{d^k}{dx^k}, D' \equiv D) $:
\begin{equation}
\begin{array}{c}
D^k\Omega_r u = \left\{
\begin{array}{l}
D^k\Omega_{2r}u, x \in [0,1/2], \\
D^k\Omega_{1r}u, x \in [1/2,1],
\end{array}
\right.
k=1,2,...
\end{array}
\end{equation}
где в соответствии с леммами 1.3~\eqref{lemma1.3} и 2.3~\eqref{lemma2.3} операторы $ D^k\Omega_{1r}u $ и $ D^k\Omega_{2r}u $ определены в формулах (3) и (9) соответственно.
\end{frame}

\begin{frame}
\label{theorem3.1}
\textbf{Теорема 3.1}
\textit{Для любой функции $ u(x) \in C^l[0,1], l \geq 0 $ выполняется сходимость:}
\begin{equation}
\begin{array}{c}
\Vert D^k\Omega_r u - u^{(k)} \Vert_{L_\infty[0,1]} \rightarrow 0 $ \textit{при} $ r \rightarrow \infty, k =0,1,...,l.
\end{array}
\end{equation}

Рассмотрим операторы $ D^m\Omega_r^{(k)} $ при $ k \geq 1, m = 0,...,k-1 $:
\begin{equation}
\begin{array}{c}
D^m\Omega_r^{(k)} u = \left\{
\begin{array}{l}
D^m\Omega_{2r}^ku \equiv r^k\int\limits_x^1 K_{2m}(x,t,k,r) u(t)dt, x \in [0,1/2], \\
D^m\Omega_{1r}^ku \equiv r^k\int\limits_0^x K_{1m}(x,t,k,r) u(t)dt, x \in [1/2,1].
\end{array}
\right.
\end{array}
\end{equation}
где $ K_{1m}(x,t,k,r) $, $ K_{2m}(x,t,k,r) $ определены в формулах (1.22),(2.19) в соответствии с леммами 1.8~\eqref{lemma1.8} и 2.8~\eqref{lemma2.8}.

\label{theorem3.2}
\textbf{Теорема 3.2}
\textit{Для любой функции $ u(x) \in C^{k-1}[0,1] $ при $ k \geq 1, m = 0,...,k-1 $ выполняется сходимость:}
\begin{equation}
\begin{array}{c}
\Vert D^m\Omega_r^{(k)} u - u^{(m)} \Vert_{L_\infty[0,1]} \rightarrow 0 $ \textit{при} $ r \rightarrow \infty .
\end{array}
\end{equation}
\end{frame}

\begin{frame}
\frametitle{Задача восстановления функции}
\label{theorem3.3}
\textbf{Теорема 3.3}
\textit{Для сходимости}
\begin{equation}
\begin{array}{c}
\Delta(\delta, \Omega_r, u) \equiv \sup\limits_{f_\delta} \lbrace \Vert \Omega_r f_\delta - u \Vert_{L_\infty[0,1]}: \Vert f_\delta - u \Vert_{L_2[0,1]} \leq \delta \rbrace \rightarrow 0 $ \textit{при} $ \delta \rightarrow 0
\end{array}
\end{equation}
\textit{необходимо и достаточно выполенение согласования:}
\begin{equation}
\begin{array}{c}
\nonumber
r = r(\delta), r(\delta) \rightarrow \infty, (r(\delta))^{1/2}\delta \rightarrow 0.
\end{array}
\end{equation}
\end{frame}

\begin{frame}
\frametitle{Задача восстановления производной функции порядка $ m $}
\label{theorem3.4}
\textbf{Теорема 3.4}
\textit{Для сходимости}
\begin{equation}
\begin{array}{c}
\Delta(\delta, D^m\Omega_r^{(k)}, u) \equiv \\ \equiv \sup\limits_{f_\delta} \lbrace \Vert D^m\Omega_r^{(k)} f_\delta - u^{(m)} \Vert_{L_\infty[0,1]}: \Vert f_\delta - u \Vert_{L_2[0,1]} \leq \delta \rbrace \rightarrow 0 $ \textit{при} $ \\
\delta \rightarrow 0, k \geq 1, 0 \leq m \leq k-1,
\end{array}
\end{equation}
\textit{необходимо и достаточно выполенение согласования:}
\begin{equation}
\begin{array}{c}
\nonumber
r = r(\delta), r(\delta) \rightarrow \infty, (r(\delta))^{\frac{2m+1}{2}}\delta \rightarrow 0.
\end{array}
\end{equation}
\end{frame}

\end{document}