\documentclass[a4paper,14pt,russian]{extreport}

\hoffset 0pt
\voffset 0pt

\usepackage[
  a4paper, includehead, includefoot, mag=1000,
  headsep=0mm, headheight=0mm,
  left=25mm, right=15mm, top=20mm, bottom=20mm
]{geometry}

\usepackage[T2A]{fontenc}
\usepackage[utf8]{inputenc}
\usepackage[russian]{babel}

% \usepackage{cmap} % Улучшенный поиск русских слов в полученном pdf-файле
\usepackage[unicode, pdftex]{hyperref}
\usepackage{pdfpages}
\usepackage[nottoc]{tocbibind}

\usepackage[onehalfspacing]{setspace} %"умное" расстояние между строк - установить 1.5 интервала от нормального
\usepackage{cite}  %"умные" библиографические ссылки (сортировка и сжатие)
\usepackage{indentfirst} %делать отступ в начале параграфа
\usepackage{enumerate}  %создание и автоматическая нумерация списков
\usepackage{longtable} % Длинные таблицы
\usepackage{multirow,makecell,array} % Улучшенное форматирование таблиц
\usepackage{graphicx} \graphicspath{{images/}}
\usepackage{pdflscape} % Для включения альбомных страниц
\renewcommand{\rmdefault}{ftm} % Включаем Times New Roman
%%% Выравнивание и переносы %%%
\sloppy % Избавляемся от переполнений
\clubpenalty=10000 % Запрещаем разрыв страницы после первой строки абзаца
\widowpenalty=10000 % Запрещаем разрыв страницы после последней строки абзаца
\righthyphenmin=2 % Минимальное число символов при переносе - 2.

\usepackage{fancyvrb}

\usepackage{amssymb,amsmath,amsfonts,latexsym,mathtext} %расширенные наборы  математических символов

\usepackage{amsthm}
\theoremstyle{definition}
\newtheorem{theorem}{Теорема}
\newtheorem{proposition}[theorem]{Предложение}
\newtheorem{corollary}[theorem]{Следствие}
\newtheorem{lemma}[theorem]{Лемма}
\newtheorem{definition}[theorem]{Определение}
\newtheorem{example}[theorem]{Пример}
\newtheorem{remark}[theorem]{Замечание}

\usepackage[tableposition=top]{caption}
\DeclareCaptionLabelFormat{gostfigure}{Рисунок #2}
\DeclareCaptionLabelFormat{gosttable}{Таблица #2}
\DeclareCaptionLabelSeparator{gost}{~---~}
\captionsetup{labelsep=gost}
\captionsetup[figure]{labelformat=gostfigure}
\captionsetup[table]{labelformat=gosttable}

\usepackage{fancyhdr}

\pagestyle{fancyplain}
\renewcommand{\headrulewidth}{0pt}
\fancyhf{}
\rfoot{\fancyplain{}{\thepage}}

\makeatletter 
\renewcommand\chapter{\if@openright\cleardoublepage\else\clearpage\fi \thispagestyle{fancyplain}%
\global\@topnum\z@ \@afterindentfalse \secdef\@chapter\@schapter} 
\makeatother

\addtocontents{toc}{\protect\thispagestyle{fancyplain}}

\setcounter{page}{1}

\usepackage{titlesec}
\titleformat{\chapter}
	{\normalsize\bfseries}
	{\thechapter}
	{1em}{}
	
\titleformat{\section}
	{\normalsize\bfseries}
	{\thesection}
	{1em}{}
	
\titleformat{\subsection}
	{\normalsize\bfseries}
	{\thesubsection}
	{1em}{}

\titleformat{\paragraph}
	{\normalsize\bfseries}
	{\thesection}
	{1em}{}

	
\titlespacing*{\chapter}{0pt}{-30pt}{8pt}
\titlespacing*{\section}{\parindent}{*4}{*4}
\titlespacing*{\subsection}{\parindent}{*4}{*4}
\titlespacing*{\paragraph}{\parindent}{*4}{*4}

\addto\captionsrussian{%
  \renewcommand\contentsname{CОДЕРЖАНИЕ}
  \renewcommand\appendixname{ПРИЛОЖЕНИЕ}
  \renewcommand\bibname{СПИСОК ИСПОЛЬЗОВАННЫХ ИСТОЧНИКОВ}
}

\usepackage{enumitem}
\makeatletter
    \AddEnumerateCounter{\asbuk}{\@asbuk}{м)}
\makeatother
\setlist{nolistsep}
\renewcommand{\labelitemi}{-}
\renewcommand{\labelenumi}{\asbuk{enumi})}
\renewcommand{\labelenumii}{\arabic{enumii})}

\usepackage{tocloft}
\renewcommand{\cfttoctitlefont}{\hspace{0.38\textwidth} \bfseries\MakeUppercase}
\renewcommand{\cftbeforetoctitleskip}{-1em}
\renewcommand{\cftaftertoctitle}{\mbox{}\hfill \\ \mbox{}\hfill{\footnotesize Стр.}\vspace{-2.5em}}
\renewcommand{\cftchapfont}{\normalsize\bfseries}
\renewcommand{\cftsecfont}{\hspace{31pt}}
\renewcommand{\cftsubsecfont}{\hspace{11pt}}
\renewcommand{\cftbeforechapskip}{1em}
\renewcommand{\cftparskip}{-1mm}
\renewcommand{\cftdotsep}{1}
\renewcommand{\cftchapdotsep}{\cftdotsep}
\setcounter{tocdepth}{2} % задать глубину оглавления — до subsection включительно

\newcommand{\likechapterheading}[1]{
    \newpage
    \begin{center}
    \textbf{\MakeUppercase{#1}}
    \end{center}}

\newcommand{\abbreviations}{\likechapterheading{ОБОЗНАЧЕНИЯ И СОКРАЩЕНИЯ}\addcontentsline{toc}{chapter}{ОБОЗНАЧЕНИЯ И СОКРАЩЕНИЯ}}
\newcommand{\definitions}{\likechapterheading{ОПРЕДЕЛЕНИЯ}\addcontentsline{toc}{chapter}{ОПРЕДЕЛЕНИЯ}}
\newcommand{\abbrevdef}{\likechapterheading{ОПРЕДЕЛЕНИЯ, ОБОЗНАЧЕНИЯ И СОКРАЩЕНИЯ}\addcontentsline{toc}{chapter}{ОПРЕДЕЛЕНИЯ, ОБОЗНАЧЕНИЯ И СОКРАЩЕНИЯ}}
\newcommand{\intro}{\likechapterheading{ВВЕДЕНИЕ}\addcontentsline{toc}{chapter}{ВВЕДЕНИЕ}}
\newcommand{\conclusions}{\likechapterheading{ЗАКЛЮЧЕНИЕ}\addcontentsline{toc}{chapter}{ЗАКЛЮЧЕНИЕ}}

\makeatletter
  \renewcommand{\@biblabel}[1]{#1.}	% Заменяем библиографию с квадратных скобок на точку:
\makeatother

\newcommand{\biblio}{
  \bibliographystyle{ugost2008}	% Оформляем библиографию в соответствии с ГОСТ 7.0.5
  \nocite{*}
  \bibliography{biblio}
}

\newcommand{\appendxchapter}[1]{ 
    \clearpage
    \stepcounter{chapter}    
    \chapter*{\appendixname~\Asbuk{chapter}\;#1}
    \addcontentsline{toc}{chapter}{\appendixname~\Asbuk{chapter}\;#1}
}
